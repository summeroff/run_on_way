\documentclass[12pt,a4paper]{book}
\usepackage{fontspec}
\usepackage{polyglossia}
\setmainfont{Noto Serif}
\setmainlanguage{russian}
\setotherlanguage{english}

\usepackage{microtype}            % Improves typography (e.g., spacing, kerning)

% Page layout
\usepackage[a4paper,margin=2.5cm]{geometry} % Consistent margins

% Graphics and images
\usepackage{graphicx}             % For including images if needed

% Headers and footers
\usepackage{fancyhdr}
\pagestyle{fancy}
\fancyhf{}                        % Clear default headers/footers
\lhead{\nouppercase{\leftmark}}   % Chapter title in header (no uppercase)
\rhead{\thepage}                  % Page number on right

% Hyperlinks and clickable references
\usepackage[hidelinks]{hyperref}  % Hyperlinks without visible boxes

\setlength{\headheight}{14.5pt} % Fix fancyhdr warning
\pagestyle{fancy}

% Custom environments
\newenvironment{dialogue}{\begin{quote}\itshape\begin{itemize}\item[]}{\end{itemize}\end{quote}}

% Verse spacing adjustment (optional, for better poetry rendering)
\usepackage{setspace}
\AtBeginEnvironment{verse}{\setstretch{1.2}} % Slightly wider line spacing in verse

\newenvironment{innerthought}{\begin{quote}\small\itshape}{\end{quote}}

\usepackage{epigraph}

% Document metadata
\title{Пробежать свой путь (Run Your Way)}
\author{SummerOfF}
\date{\today}

\begin{document}
\maketitle
\tableofcontents
% Chapters will follow here...

\chapter{Безмолвный наблюдатель}
\epigraph{«Толпа --- это сила, лишённая разума.»}{Гюстав Лебон}
\begin{verse}
Не лает, не кусается,\\
На кошек не бросается,\\
И на прохожих ноль внимания,\\
Такое вот у нас воспитание.
\end{verse}

Эти строки, словно заезженная пластинка, крутились в голове Алексея каждое утро. Они были его мантрой, щитом от внешнего мира --- мира, который он предпочитал держать на расстоянии. Алексей --- высокий, худощавый, с кожей такой бледной, что она казалась почти прозрачной, и тёмными глазами, где пряталась неуловимая тень. Его взгляд редко цеплялся за что-то конкретное, ускользая куда-то за горизонт, будто он искал там ответы, недоступные остальным. Лицо его было спокойным, застывшим, как маска, но за этой маской угадывалась глубина, которую он никому не показывал.

\section{Утро как ритуал}

День начинался ровно в 7:00. Не было будильника --- тело само знало, когда просыпаться. Комната Алексея напоминала монашескую келью: узкая кровать с серым покрывалом, деревянный стол, стул да полка, плотно уставленная книгами. На стене висел единственный постер --- лабиринт, чьи чёрные линии извивались, точно отражая его мысли. Никаких фотографий, никаких безделушек --- только книги: психология, философия, исследования человеческого поведения. Они были его окнами в мир, который он не решался потрогать руками.

Он одевался быстро, без лишних движений: серые брюки, белая рубашка, чёрный пиджак --- цвета, которые растворялись в толпе. Алексей не хотел выделяться. Быть замеченным значило стать уязвимым, а уязвимость он презирал, хоть и не признавался в этом даже себе.

Улица встретила его привычным гулом: машины ревели, люди спешили, голоса переплетались в утренний шумовой коктейль. Но он шёл своей дорогой, не поднимая глаз, не отвечая на редкие \textit{«доброе утро»} соседей. Его шаги были размеренными, почти механическими, а лицо оставалось непроницаемым --- стена между ним и хаосом вокруг.

\section{День без волн}

В метро Алексей занял своё место у окна --- не потому, что любил вид, а чтобы отгородиться от чужих локтей и взглядов. Он смотрел на чёрные стены туннеля, на мелькающие отражения пассажиров в стекле. Его собственное отражение было размытым, призрачным пятном, и это его устраивало. Разговоры вокруг --- обрывки сплетен, смех, жалобы --- скользили мимо, не задевая. Он не вмешивался, не реагировал, просто наблюдал, как учёный за стеклом лаборатории.

Работа в IT-компании была такой же серой, как его одежда. Алексей чинил баги, отвечал на запросы клиентов, молчал. Коллеги считали его странным, но безобидным --- «Тихий», шептались они за его спиной, и он ловил эти слова с лёгкой, скрытой улыбкой. Тишина была его союзником, его способом выжить среди людей, которые не понимали, зачем жить незаметно.

В обед он уходил в парк. Садился на скамейку с книгой --- сегодня это была социальная психология, о том, как общество лепит личность, подчиняет её незаметными нитями. Иногда он поднимал глаза: мама с коляской, старик с газетой, подростки с телефонами. Они казались ему марионетками в чужой пьесе, а он --- зрителем, который отказался играть свою роль.

\section{Трещина в привычном}

Вечер начался как обычно. Алексей возвращался домой, шаги отмеряли знакомый ритм, пока не раздался гул. Его улица, обычно тихая, бурлила. Толпа собралась у одного из домов --- голоса сливались в тревожный хор, лица мелькали в свете фонарей. Он замедлил шаг, но не остановился. Толпы были его врагом --- хаотичные, непредсказуемые, полные чужих эмоций. Он не хотел быть частью этого.

Проходя мимо, он уловил обрывки:

\begin{dialogue}
«--- \ldots такой молодой\ldots»\\[0.5em]
«--- \ldots никто не ожидал\ldots»\\[0.5em]
«--- \ldots как это вообще могло случиться?»
\end{dialogue}

Его брови дрогнули --- едва заметное движение, выдавшее тень любопытства. Что-то случилось, но он не стал спрашивать. Подобное вторжение в чужую жизнь противоречило его правилам. Он ускорил шаг, уходя к своему подъезду.

Дома он приготовил ужин --- макароны с сыром, простое блюдо, не требующее мыслей. Телевизор включился для фона, но вскоре его голос прорвался сквозь привычный шум. Репортаж: его улица, мигалки, толпа. Диктор говорил сухо: «Молодой человек найден мёртвым. Предположительно --- самоубийство. Расследование продолжается.»

Алексей замер, вилка зависла в воздухе. «Самоубийство?» --- слово ударило, как холодный ветер. Он вспомнил лица в толпе, их шёпот. Это был сосед? Кто-то, мимо кого он проходил каждый день, не замечая? Мысль кольнула, но он отогнал её, выключив телевизор. Тишина вернулась, но теперь она была другой --- тяжёлой, настороженной.

\section{Ночь вопросов}

Он подошёл к окну. Улица опустела --- ни машин, ни людей, только жёлтые пятна фонарей на асфальте. Но в груди росло что-то новое --- тревога? Нет, скорее зуд, желание понять. Он не мог назвать это точно.

Сев за ноутбук, он открыл новости. Статьи были скупы: «молодой человек», «обнаружен соседями», «причины неизвестны». Ничего, что могло бы зацепиться в памяти. Алексей откинулся на стуле, глядя на полку. Его рука потянулась к книге --- «Психология толпы» Лебона. Он открыл её наугад:

\begin{quote}
Толпа --- это сила, лишённая разума. Она движется импульсами, подчиняясь страху или восторгу, не ведая, куда идёт.
\end{quote}

Толпа на улице. Что ими управляло? Скорбь? Жажда сплетен? Или что-то глубже, скрытое под обыденностью? Вопросы кружились, не давая покоя.

\section{Шаг к пропасти}

Он лёг в кровать, но сон не шёл. Мысли возвращались к случившемуся. Перед глазами всплыла бабушка --- её строгая фигура, её голос:

\begin{innerthought}
«Не вмешивайся, Алексей. Не привлекай внимания. Живи тихо.»
\end{innerthought}

Эти слова он впитал с детства, после смерти родителей, когда она стала его единственным проводником в мире. Они были его законом, его бронёй. Но сегодня броня дала трещину. Впервые он ощутил, что не может просто отвернуться. Что-то тянуло его узнать больше --- не ради жалости, а ради самого себя.

«Завтра поговорю с соседями», --- решил он, закрывая глаза. Или хотя бы найду больше информации. Но в темноте перед ним вставали образы: толпа, мигалки, чьё-то лицо, которое он никогда не видел. Он не знал, что это начало конца его тихой жизни, что силы, которые он избегал, уже заметили его --- и теперь не отпустят.

\chapter{Точка разлома}

Алексей сидел в своей комнате, уставившись на пустой экран ноутбука. Тишина обволакивала его, знакомая, как старый свитер, но сегодня она не приносила утешения. Он только что закончил очередной рабочий день --- тихо, незаметно устраняя баги в коде для компании, где его знали только как «Тихого». Это прозвище, шепотом произнесённое коллегами за его спиной, было его бронёй, его способом остаться невидимым. Но теперь броня трещала по швам.

Случай с самоубийством на его улице — парень, найденный мёртвым в соседнем доме, — изменил всё. Тишина, прежде его союзник, теперь звенела в ушах, натянутая, как струна на грани разрыва. Алексей встал и подошёл к окну. Улица лежала перед ним пустая, освещённая лишь миганием фонарей, чьи тени плясали на асфальте, словно призраки. Он смотрел на них, и внутри что-то шевельнулось — не тревога, а холодное, острое предчувствие.

\section{Трещина растёт}

Утром он решился нарушить своё правило невмешательства. Спустившись к соседям, он постучался в дверь напротив — старую, облупившуюся, с потёртым номером «12». Её открыла пожилая женщина, чьи глаза, усталые и выцветшие, скользнули по его лицу, будто взвешивая.

\begin{dialogue}
«— Вы слышали про парня?» — спросил он, стараясь держать голос ровным, нейтральным.
\end{dialogue}

Женщина кивнула, её губы дрогнули в слабой, вымученной улыбке.

\begin{dialogue}
«— Бедняга. Никто не ожидал. Говорят, он был странным… замкнутым. А потом просто… сломался.»
\end{dialogue}

\begin{dialogue}
«— Сломался?» — переспросил Алексей. Слово ударило, как камешек о стекло, оставив трещину в его спокойствии.
\end{dialogue}

\begin{dialogue}
«— Да», — она вздохнула, глядя куда-то мимо него. — «Люди судачили, что он не выдержал. Знаешь, как это бывает — все смотрят, все шепчутся. А он… он не смог.»
\end{dialogue}

Она замолчала, а Алексей почувствовал, как холод пробежал по спине. Он знал это чувство — взгляды, обжигающие затылок, слова, проникающие под кожу, как иглы. Поблагодарив её сухим «спасибо», он ушёл, но слова женщины застряли в голове: «Все смотрят, все шепчутся». Они звучали как эхо его собственной жизни, и от этого становилось не по себе.

\section{Эхо чужой боли}

Дома он включил телевизор, чтобы заглушить мысли, но вместо фона экран выдал тот же репортаж. Его улица, мигалки, голос диктора: «Давление общества… записка: ‘Я больше не могу их слышать’». Алексей выключил телевизор резким движением, пальцы дрожали на пульте. Записка. Эти слова — «не могу их слышать» — ввинтились в его сознание, как шурупы. Он и сам порой чувствовал эти взгляды: в метро, на работе, даже среди соседей. Они были повсюду, невидимая сеть, затягивающая всё туже.

Он попытался отвлечься, но мысли возвращались к парню. Кто он был? Может, такой же, как Алексей — тень, растворённая в толпе, пока давление не раздавило его? От этой мысли становилось душно, будто стены комнаты сжимались.

\section{Давление улицы}

На следующий день он решил прогуляться — проветрить голову, вытряхнуть из неё чужую смерть. Но улицы остались прежними: шумными, суетливыми, набитыми людьми, которые либо не замечали его, либо смотрели слишком долго. Проходя мимо кафе, он поймал взгляды подростков — их громкий смех резанул по нервам, а глаза будто буравили его. Старик на скамейке пробормотал что-то, глядя ему вслед, и Алексей ускорил шаг. Каждый звук — шорох шин, обрывок разговора — казался громче, острее, как будто мир решил напомнить ему, что он здесь чужой.

Когда он вернулся домой, его ждал сюрприз. На дверной ручке висела записка — кривой почерк, чёрные чернила: 

\begin{quote}
«Ты следующий.»
\end{quote}

Алексей замер, рука застыла в воздухе. Он оглянулся — улица была пуста, только ветер гнал листок бумаги по асфальту. Сорвав записку, он скомкал её, но слова уже врезались в память, как раскалённый металл. Это не шутка. Это предупреждение. Сердце заколотилось, и впервые за долгое время он почувствовал не просто тревогу — страх, настоящий, липкий, сжимающий горло.

\section{Ночь без сна}

Он почти не спал. Лёжа в темноте, прислушивался к каждому шороху за окном: шаги? Шёпот? Или просто ветер? Мысли кружились, выстраивая цепочки: самоубийство, толпа, записка. Всё складывалось в мрачную картину, и он был в её центре — не зритель, а мишень. Он ворочался, пытаясь прогнать видения: пустые глаза соседа, толпа, смыкающаяся вокруг него, чья-то тень за дверью.

Утром он принял решение. Этот город, эти люди, эта жизнь — всё это медленно убивало его, как того парня. Он больше не мог оставаться тенью под их взглядами. Нужно бежать, начать заново, где-то, где его не найдут, где он сможет дышать.

\section{Побег}

Алексей открыл ноутбук и начал искать. Через час он наткнулся на объявление: небольшой дом в тихом городке у реки, вдали от шума и любопытных глаз. Цена смешная, фотографии обещали покой — деревянная веранда, зелёные холмы, река, сверкающая на солнце. Он забронировал его, не раздумывая, пальцы стучали по клавишам с непривычной поспешностью.

Собирая вещи, он чувствовал странное облегчение. Книги легли в коробку, одежда — в чемодан. 

\begin{innerthought}
«Я уйду от всего этого», — говорил он себе, закрывая полки. — «Найду место, где смогу быть собой.»
\end{innerthought}

Но в глубине души он знал: от себя не убежать. Трещина, что появилась после смерти соседа, росла, и побег был лишь попыткой её замазать.

Когда он закрыл дверь квартиры в последний раз, то бросил взгляд на улицу. Ему показалось, что в тени дома напротив кто-то стоит — тёмная фигура, неподвижная, смотрящая прямо на него. Алексей стиснул ручку чемодана, но не обернулся. Сев в машину, он включил двигатель и поехал прочь, оставляя позади город, который начал его ломать. Дорога впереди была пустой, но тень в зеркале заднего вида осталась с ним — невидимая, но ощутимая.

\chapter{Тихая гавань}

\begin{verse}
Не лает, не кусается,\\
На кошек не бросается,\\
И на прохожих ноль внимания,\\
Такое вот у нас воспитание.
\end{verse}

Эти строки эхом звучали в голове Алексея, пока он шагал по узкой тропинке к своему новому дому. Они были его прошлым — кодом, который он оставил позади, уезжая из города, где давление чужих взглядов чуть не раздавило его. Теперь он здесь, в маленьком городке, затерянном среди зелёных холмов, — месте, которое обещало покой и шанс начать заново. После всего — записок, смерти соседа, тени в зеркале — ему нужна была тишина, чтобы зашить трещины в своей душе.

Алексей остановился у ворот дома — скромного, но уютного, с деревянной верандой и видом на реку, что блестела вдали, как серебряная нить. Воздух был свежим, пропитанным запахом сосен и влажной земли. Он вдохнул глубоко, чувствуя, как напряжение последних недель начинает растворяться, хотя тень тревоги всё ещё цеплялась за края сознания.

\begin{dialogue}
«— Вот и всё», — пробормотал он, толкая калитку. — «Новая жизнь.»
\end{dialogue}

\section{Первые шаги}

Двор встретил его тишиной, нарушаемой лишь шелестом листвы. Дом был таким, как на фото: два этажа, белые ставни, лужайка с цветущими кустами. Алексей улыбнулся — впервые за долгое время, — представив, как будет сидеть на веранде с кофе, слушая пение птиц вместо городского гула.

Из-за угла выскочил пёс — крупный, лохматый, с шерстью цвета ржи и глазами, полными добродушного любопытства. Он подбежал к Алексею, виляя хвостом, и ткнулся мокрым носом в его ладонь.

\begin{dialogue}
«— Привет, дружище», — сказал Алексей, гладя пса по голове. — «Ты, похоже, местный?»
\end{dialogue}

Пёс лизнул его руку и уселся рядом, будто они уже были старыми знакомыми. Алексей рассмеялся — звук вырвался неожиданно, легко, как будто кто-то снял камень с его груди.

\begin{dialogue}
«— Ладно, будешь моим компаньоном», — решил он, открывая дверь дома.
\end{dialogue}

Внутри пахло деревом и пылью — запах забытого уюта. Светлые занавески колыхались от сквозняка, деревянная мебель стояла ровно, а на стенах висели простые картины: река, лес, закат. Алексей поставил чемодан у порога и прошёлся по комнатам, впитывая тишину нового пристанища.

\section{Тень на пороге}

В спальне на втором этаже он заметил старый письменный стол, заваленный бумагами. Видимо, прежний жилец был писателем или журналистом — стопка листов, пожелтевших от времени, лежала в беспорядке. Алексей подошёл ближе и взял верхний лист. Крупными буквами было выведено: «Дневник».

Он замер, рука дрогнула. Любопытство боролось с осторожностью, но первое победило.

\begin{innerthought}
«— Может, это поможет понять это место», — подумал он, открывая страницу.
\end{innerthought}

\begin{quote}
«День первый. Я приехал сюда, чтобы забыть. Чтобы исцелиться. Но что-то здесь не так. Люди… они смотрят на меня странно. Как будто знают то, чего не знаю я.»
\end{quote}

Алексей нахмурился. Почерк был торопливым, с помарками, будто писавший нервничал. Он перевернул лист.

\begin{quote}
«День третий. Сегодня ко мне приходила соседка. Принесла пирог. Она улыбалась, но её глаза… они были пустыми. Как у куклы. Я не могу избавиться от ощущения, что за мной следят.»
\end{quote}

Сердце стукнуло сильнее. Он отложил дневник, чувствуя, как холодок пробежал по спине. «Просто параноик», — сказал он себе, но голос прозвучал неуверенно. Слишком знакомо это звучало — взгляды, слежка, давление. Он прогнал мысль и спустился на кухню.

\section{Иллюзия покоя}

Заварив чай, Алексей выглянул в окно. Солнце клонилось к закату, окрашивая небо в багровые тона. Улица была тихой — только лай собак вдали да шорох ветра. Он отхлебнул горячий чай, стараясь сосредоточиться на простых вещах: завтра он распакует вещи, может, начнёт огород. Надо купить блокнот — вести свой дневник, следить за собой, чтобы не скатиться в старую тьму.

Но тут он заметил движение у забора. Присмотревшись, увидел фигуру — человека в тёмной одежде, неподвижного, словно статуя. Алексей замер, чашка дрогнула в руке. Фигура стояла, глядя — или ему показалось? — прямо на дом. Несколько секунд, и она скрылась за деревьями, быстрым, почти бесшумным шагом.

\begin{dialogue}
«— Чёрт», — прошептал он, отходя от окна.
\end{dialogue}

Сердце заколотилось, но он заставил себя дышать ровно. «Может, прохожий», — подумал он, хотя в это не верилось. Он проверил замки, закрыл ставни, но тревога осталась — тонкая, как паутина, облепившая мысли.

\section{Ночь без границ}

Лёжа в кровати, Алексей ворочался, прислушиваясь к шорохам. Новый дом скрипел, как старый корабль, — то ли от ветра, то ли от чего-то ещё. Ему показалось, что он слышит шаги за дверью, лёгкие, крадущиеся. Он встал, сжал кулаки и открыл дверь спальни. Пусто. Только тьма и слабый свет луны, пробивавшийся сквозь щели.

Пёс — он решил звать его Барон — поднял голову, но не залаял, лишь смотрел на хозяина с немым вопросом. Алексей выдохнул, укоряя себя за нервы. Он вернулся в кровать, но сон не шёл. Перед глазами вставали строки из дневника: «За мной следят». А потом — тень у забора, неподвижная, как угроза.

Он закрыл глаза, но темнота ожила: ему мерещились шаги, шёпот, чьё-то дыхание у окна. Это был его первый день в «тихой гавани», но покой, которого он искал, оказался иллюзией. Что-то — или кто-то — уже знало о его приезде. И это что-то не собиралось оставлять его в покое.

\chapter{Светлый день}

Алексей проснулся рано утром. Солнце только начинало пробиваться сквозь занавески, заливая комнату мягким золотистым светом. Он потянулся, чувствуя себя отдохнувшим, несмотря на тревожные мысли, которые одолевали его накануне вечером. «Сегодня будет хороший день»,~-- сказал он себе, поднимаясь с кровати.

Спустившись на кухню, он заварил кофе и вышел на веранду. Его новый спутник~-- пёс, которого он решил назвать Барон,~-- уже ждал его, радостно виляя хвостом. Алексей улыбнулся и потрепал его по голове.

\begin{dialogue}
«-- Доброе утро, дружище. Готов к новому дню?»
\end{dialogue}

Барон коротко тявкнул, словно соглашаясь, и Алексей рассмеялся. Он присел на ступеньки веранды, вдыхая свежий утренний воздух. Вдалеке журчала река, а вокруг раздавалось пение птиц. Это было именно то, чего он искал, переезжая сюда~-- покой и уединение.
После завтрака Алексей решил прогуляться по городку. Ему хотелось получше узнать место, куда он переехал, а заодно познакомиться с соседями. Надев лёгкую куртку, он вышел на улицу.

Городок оказался небольшим, но удивительно уютным. Чистые улочки, ухоженные дома и дружелюбные лица прохожих создавали ощущение тепла. Проходя мимо продуктового магазина, Алексей заметил мужчину средних лет, который раскладывал фрукты у входа. Тот поднял голову и, заметив Алексея, приветливо улыбнулся.

\begin{dialogue}
«-- Привет! Новенький в городе, да?» — спросил мужчина, вытирая руки о фартук.
\end{dialogue}

\begin{dialogue}
«-- Привет. Да, вчера только приехал. Меня зовут Алексей.»
\end{dialogue}

\begin{dialogue}
«-- Рад познакомиться, Алексей! Я Миша, держу этот магазин. Если что понадобится, заходи, не стесняйся.»
\end{dialogue}

Алексей кивнул и продолжил прогулку. Люди на улице здоровались с ним, улыбались, некоторые даже останавливались, чтобы обменяться парой слов. Он познакомился с местным пекарем, который угостил его свежим хлебом, и с библиотекарем, которая пригласила заглянуть в библиотеку.

К середине дня Алексей почувствовал, что его тревоги начинают отступать. «Может, я зря волновался»,~-- подумал он, вернувшись домой и устроившись с чашкой чая за столом.~-- «Люди здесь добрые, и мне здесь нравится.»

Но вечером, когда он готовил ужин, его мысли вновь вернулись к той странной записке, которую он нашёл вчера. Не в силах избавиться от беспокойства, он решил проверить дом. Алексей обошёл все комнаты, проверил окна и двери~-- всё было заперто. Однако, войдя в спальню, он замер. На кровати лежала ещё одна записка, написанная тем же неровным почерком: «Ты не спрячешься».

Его руки задрожали. Схватив записку, он скомкал её и швырнул в угол комнаты. «Кто это делает? Зачем?»~-- выкрикнул он в пустоту. Барон, почуяв его волнение, заскулил и прижался к его ноге.

\begin{dialogue}
«-- Всё в порядке, дружище»,~-- сказал Алексей, стараясь успокоить себя больше, чем собаку.~-- «Может, это просто детская шалость»,~-- подумал он, хотя в глубине души понимал, что это не так.
\end{dialogue}

Он решил позвонить в полицию. Набрав номер местного участка, Алексей рассказал о записках. Дежурный офицер выслушал его и ответил спокойным голосом:

\begin{dialogue}
«-- Не волнуйтесь, мы разберёмся. Но, скорее всего, это просто чья-то глупая шутка.»
\end{dialogue}

Повесив трубку, Алексей почувствовал небольшое облегчение. Он решил быть начеку, но не позволить этому испортить его новую жизнь.

\chapter{Остров спасения}

Утро началось с солнца, льющегося через окно прямо на лицо Алексеya. Он открыл глаза, чувствуя, как тепло растекается по телу. Впервые за несколько дней он выспался. Записки, тревоги, ночные шорохи~-- всё это казалось далёким сном. Барон, свернувшись калачиком у кровати, тихо посапывал.

\begin{dialogue}
    «-- Пора встряхнуться», -- сказал Алексей, вставая.
\end{dialogue}

Он решил, что хватит сидеть в четырёх стенах и ждать, пока страх сожрёт его изнутри. Сегодня он попробует жить, как нормальный человек.

После завтрака он позвонил Мише, парню из магазина, с которым успел обменяться парой слов накануне.

\begin{dialogue}
«-- Привет, Миш. Не хочешь на рыбалку? У реки вроде красиво, да и погода отличная.»
\end{dialogue}

\begin{dialogue}
«-- Давай! Бери удочки, если есть, а я захвачу пиво. Встретимся у моста через час.»
\end{dialogue}

Алексей улыбнулся. Может, это и есть то, ради чего он сюда приехал~-- простые радости, дружеская компания, возможность дышать полной грудью.

Он собрал небольшой рюкзак: удочку, найденную в сарае, бутерброды, бутылку воды. Барон проводил его до двери, но Алексей решил оставить пса дома~-- на рыбалке тот мог бы спугнуть улов.

У реки было тихо. Вода блестела под солнцем, лёгкий ветерок шевелил листву. Миша уже ждал у моста, держа в руках пару банок пива и старый рюкзак.

\begin{dialogue}
«-- Ну что, новичок, готов поймать свою первую рыбу?»
\end{dialogue}

\begin{dialogue}
«-- Если только она сама на крючок прыгнет»,~-- пошутил Алексей, и оба рассмеялись.
\end{dialogue}

Они нашли укромное место на берегу, где река делала изгиб, и разложили снасти. Миша оказался весёлым собеседником: травил байки про местных жителей, рассказывал, как однажды поймал щуку размером с руку, и даже спел пару строчек из какой-то дурацкой песни. Алексей расслабился, чувствуя, как напряжение последних дней уходит. Они пили пиво, шутили, а удочки покачивались в воде, словно соглашаясь с их беззаботностью.

\begin{dialogue}
«-- Знаешь»,~-- сказал Миша, глядя на реку,~-- «тут хорошо. Спокойно. Иногда кажется, что весь мир где-то там, а мы тут в своём маленьком раю.»
\end{dialogue}

\begin{dialogue}
«-- Да»,~-- кивнул Алексей,~-- «именно этого мне и не хватало.»
\end{dialogue}

Солнце поднялось выше, и они решили перебраться на лодку, которую Миша притащил из кустов. Это была старая деревянная посудина, но выглядела крепкой. Они оттолкнулись от берега, и река мягко понесла их вниз по течению. Алексей наблюдал, как деревья проплывают мимо, как отражаются облака в воде, и впервые за долгое время чувствовал себя свободным.

Но к вечеру небо начало темнеть. Облака сгустились, ветер усилился, и река, ещё недавно спокойная, заволновалась. Миша нахмурился.

\begin{dialogue}
«-- Похоже, буря идёт. Надо возвращаться.»
\end{dialogue}

Они начали грести к берегу, но волны становились всё выше. Лодка качалась, вода плескалась через борта. Алексей сжал вёсла, стараясь держать ритм, но сердце уже билось быстрее.

\begin{dialogue}
«-- Держись!»~-- крикнул Миша, когда очередной порыв ветра накренил лодку.
\end{dialogue}

И тут раздался треск. Днище треснуло, и холодная вода хлынула внутрь. Алексей бросил вёсла, хватаясь за борта, но лодка стремительно тонула. Они оказались в воде, среди темноты и воющего ветра.

\begin{dialogue}
«-- Миша!»~-- крикнул Алексей, пытаясь разглядеть друга в хаосе волн.
\end{dialogue}

Он увидел его в нескольких метрах~-- Миша барахтался, но его лицо\ldots Алексей замер. Миша смотрел на него с улыбкой: не с паникой, не с отчаянием, а с какой-то странной, почти зловещей улыбкой. А потом волна накрыла его, и он исчез.

Алексей боролся с течением, лёгкие горели, руки немели от холода. Он не знал, сколько времени прошло, но наконец его выбросило на берег. Кашляя и задыхаясь, он рухнул на мокрую траву. Дождь хлестал по лицу, молнии освещали небо.

Он лежал, пытаясь отдышаться, когда заметил что-то в кармане куртки. Дрожащими руками он вытащил мокрый клочок бумаги. На нём размытыми чернилами было написано:
\begin{quote}
«Мы ближе, чем ты думаешь.»
\end{quote}

Алексей отбросил записку, его разум закружился. Миша? Это он? Или кто-то другой? Дождь смывал грязь с его рук, но не мог смыть страх, который теперь жил внутри него. Он поднялся и побрёл домой, оставляя за собой реку, которая чуть не стала его могилой.

Когда он дошёл до дома, Барон встретил его тревожным лаем. Алексей рухнул на пол, прижимая пса к себе. Его тело дрожало, а в груди росло ощущение, что даже здесь, в этом «раю», он не один. Кто-то~-- или что-то~-- следило за ним. И этот кто-то знал, как добраться до него.

\chapter{Тепло очага}

Алексей проснулся от лая Барона. Солнечный свет пробивался через щели в занавесках, но он чувствовал себя так, будто не спал вовсе. Ночь после рыбалки была полна кошмаров: тёмная вода, ухмылка Миши, записка, растворяющаяся в реке. Он сел на кровати, потирая виски. Тело ломило от вчерашнего купания в холодных волнах, но хуже всего было внутри~-- страх, как ржавчина, разъедал его изнутри.

\begin{dialogue}
«-- Хватит»,~-- сказал он вслух, глядя на Барона, который тревожно смотрел на него.~-- «Я не дам этому разрушить меня.»
\end{dialogue}

Он решил взять всё в свои руки. Этот дом должен стать его убежищем, а не ловушкой. После завтрака~-- чёрный кофе и кусок хлеба, больше он не смог в себя впихнуть,~-- Алексей принялся за дело. Он проверил все замки на дверях и окнах, подкрутил шурупы там, где они болтались. В сарае он нашёл старую доску и забил ею окно на заднем дворе, которое казалось ему слишком уязвимым. Каждый стук молотка звучал как вызов: «Я не сдамся.»

К полудню дом выглядел крепче. Алексей даже позволил себе улыбнуться, глядя на Барона, который носился по двору, гоняясь за бабочкой.

\begin{dialogue}
«-- Вот так, дружище. Теперь это наша крепость.»
\end{dialogue}

Он включил старый радиоприёмник, найденный на кухне, и настроил его на местную станцию. Тихая музыка заполнила дом, создавая иллюзию уюта. Алексей сел на диван с чашкой чая, чувствуя, как напряжение понемногу отпускает. Он даже начал думать, что буря на реке~-- случайность, а записка~-- чья-то дурацкая шутка. Может, Миша просто утонул, а улыбка была плодом его воображения?

Раздался стук в дверь. Алексей вздрогнул, пролив чай на колени. Барон насторожился, но не залаял~-- просто смотрел на дверь, чуть склонив голову.

\begin{dialogue}
«-- Кто там?»~-- крикнул Алексей, не вставая.
\end{dialogue}

\begin{dialogue}
«-- Это я, Мария Ивановна!»~-- послышался знакомый голос соседки.~-- «Принесла тебе кое-что, сынок.»
\end{dialogue}

Он выдохнул, чувствуя себя немного глупо за свою реакцию. Открыв дверь, он увидел старушку с корзинкой в руках. Она улыбалась, её морщинистое лицо выглядело добрым и спокойным.

\begin{dialogue}
«-- Вот, травяной чай»,~-- сказала она, протягивая корзинку.~-- «Успокаивает нервы. Ты вчера выглядел таким бледным, подумала, тебе пригодится.»
\end{dialogue}

\begin{dialogue}
«-- Спасибо»,~-- ответил Алексей, принимая подарок.~-- «Заходите, если хотите.»
\end{dialogue}

Мария Ивановна покачала головой.

\begin{dialogue}
«-- Нет, сынок, дела зовут. Но ты пей чай, отдыхай. И заходи, если что.»
\end{dialogue}

Она ушла, а Алексей закрыл дверь, чувствуя тепло от её заботы. Он заварил чай~-- пахло мятой и чем-то ещё, терпким, но приятным. Сделав глоток, он устроился на диване, глядя, как Барон играет с костью у порога. Впервые за долгое время он почувствовал себя в безопасности.

К вечеру чай подействовал~-- веки стали тяжёлыми, тело расслабилось. Алексей лёг в кровать, даже не выключив свет на кухне. Сон пришёл быстро, глубокий и тёмный, как колодец.

Ему снилось, что он стоит посреди комнаты, а вокруг~-- люди в масках. Они не двигались, просто смотрели, их глаза блестели в темноте. Он пытался кричать, но голос пропал. Внезапно маски начали падать, и под ними были знакомые лица: Миша, соседка, даже продавец из магазина. Они улыбались, но не так, как живые люди, а как манекены.

Алексей проснулся с криком, хватая воздух. Комната была тёмной, свет на кухне давно погас. Он рванулся к выключателю, но остановился. Что-то было не так. Воздух казался тяжёлым, а тишина~-- слишком густой.

Барон зарычал, глядя на дверь спальни. Алексей медленно подошёл к ней, сердце колотилось так, что отдавалось в ушах. Он открыл дверь~-- и замер. На столе в гостиной лежала фотография. Он сам, спящий на диване, с закрытыми глазами и чуть открытым ртом. Снимок был сделан этой ночью.

\begin{dialogue}
«-- Нет\ldots нет, нет, нет»,~-- прошептал он, отступая назад.
\end{dialogue}

Барон залаял, бросаясь к окну, но там никого не было. Алексей схватил телефон, чтобы позвонить в полицию, but экран мигнул и погас~-- батарея села, хотя утром была полной.

Он рухнул на стул, сжимая голову руками. Фотография лежала перед ним, как доказательство: кто-то был здесь, внутри его «крепости», пока он спал. Чай? Неужели старушка\ldots Нет, это бред. Или не бред?

Барон продолжал рычать, и Алексей понял: покой~-- это иллюзия. Кто-то проник в его дом, в его разум, и теперь он не знал, где реальность, а где его собственные страхи.

\chapter{Дружеская поддержка}

Утро встретило Алексея холодом и тишиной. Он сидел за столом, уставившись на фотографию, которая всю ночь лежала перед ним, как немой укор. Барон спал у его ног, но даже присутствие пса не могло прогнать страх, который пустил корни в его груди. Он не спал, не ел~-- просто смотрел, пытаясь понять, кто и как это сделал.

«Я не могу так жить»,~-- наконец сказал он себе, вставая. Нужно было действовать, искать помощь. Один он с этим не справится.

После чашки кофе, которая не согрела, а только обожгла горло, Алексей решил пойти к Марии Ивановне. Её чай, её доброта~-- всё это теперь казалось подозрительным, но она была единственным человеком, с кем он хоть немного сблизился. Может, он ошибся, и она действительно просто милая старушка? Ему нужно было верить хоть кому-то.

Он постучал в её дверь, сжимая кулаки, чтобы унять дрожь. Мария Ивановна открыла почти сразу, и её лицо осветилось улыбкой.

\begin{dialogue}
«-- Ой, Алексей, заходи! Что-то ты бледный опять. Не заболел?»
\end{dialogue}

\begin{dialogue}
«-- Нет»,~-- ответил он, входя в её уютную кухню,~-- «просто\ldots не спал. Хотел поговорить.»
\end{dialogue}

\begin{dialogue}
«-- Садись, сынок»,~-- сказала она, поставив перед ним чашку чая. 
\end{dialogue}

Алексей отодвинул её, всё ещё помня вчерашний ужас.

\begin{dialogue}
«-- Мария Ивановна»,~-- начал он, стараясь говорить спокойно,~-- «у меня дома\ldots странные вещи творятся. Записки, фотографии. Кто-то был внутри, пока я спал.»
\end{dialogue}

Её глаза расширились, но не от испуга, а от любопытства. Она покачала головой.

\begin{dialogue}
«-- Боже мой, это ужасно. Ты в полицию ходил?»
\end{dialogue}

\begin{dialogue}
«-- Пока нет. Думал, может, вы что-то знаете. Видели кого-нибудь около моего дома?»
\end{dialogue}

\begin{dialogue}
«-- Нет, сынок, ничего такого. Но у нас городок тихий, знаешь\ldots Иногда детишки балуются, а иногда»~-- она замялась,~-- «старые истории оживают.»
\end{dialogue}

\begin{dialogue}
«-- Какие истории?»
\end{dialogue}

\begin{dialogue}
«-- Да ерунда, слухи. Говорят, раньше тут люди пропадали. Но это давно было, не бери в голову.»
\end{dialogue}

Алексей кивнул, но её слова только усилили тревогу. Он рассказал ей про рыбалку, про Мишу, про найденную записку. Мария Ивановна слушала внимательно, иногда качая головой, а потом положила руку на его плечо.

\begin{dialogue}
«-- Ты, главное, не бойся. Если что, приходи ко мне. Я старая, но глаза у меня зоркие. Пригляжу за твоим домом.»
\end{dialogue}

Он ушёл с лёгким облегчением. Её забота казалась искренней, и он решил дать ей шанс. Может, он правда преувеличивает? Вернувшись домой, Алексей занялся делами: почистил двор, покормил Барона, даже включил музыку, чтобы заглушить тишину. День прошёл спокойно, и к вечеру он почти поверил, что всё налаживается.

Но ночью всё рухнуло.

Он проснулся от странного ощущения~-- как будто кто-то давил ему на грудь. Открыв глаза, он увидел тень над собой. Кто-то стоял у кровати, держа подушку. Алексей рванулся, но тень прижала подушку к его лицу. Воздух исчез, он задыхался, бился, хватая руками пустоту. Наконец, он сумел сбросить нападающего, и тот с шумом упал на пол. Тень метнулась к двери и исчезла в темноте.

Алексей вскочил, кашляя и задыхаясь. Барон лаял, бросаясь к окну. Включив свет, Алексей осмотрел комнату. На полу лежала маленькая деревянная пуговица с вырезанным цветком. Он видел такую же на кофте Марии Ивановны.

\begin{dialogue}
«-- Нет\ldots»
\end{dialogue}

Сжав пуговицу в кулаке, он не мог поверить: это не могло быть правдой. Или могло?

Утром он пошёл к ней снова. Она открыла дверь, как ни в чём не бывало, с той же доброй улыбкой.

\begin{dialogue}
«-- О, Алексей, опять ты! Чайку?»
\end{dialogue}

\begin{dialogue}
«-- Нет»,~-- резко ответил он, показывая пуговицу,~-- «это ваше?»
\end{dialogue}

Она посмотрела на пуговицу, потом на свою кофту~-- действительно, одна пуговица отсутствовала. Её лицо осталось спокойным.

\begin{dialogue}
«-- Ой, наверное, потеряла где-то. А где ты её нашёл?»
\end{dialogue}

\begin{dialogue}
«-- У себя в спальне. После того, как кто-то пытался меня задушить.»
\end{dialogue}

Мария Ивановна ахнула, прижимая руку к груди.

\begin{dialogue}
«-- Господи, что ты такое говоришь? Это не я, сынок, клянусь!»
\end{dialogue}

Он смотрел ей в глаза, не зная, что думать~-- врёт ли она или нет. Её голос дрожал, но что-то в её взгляде было не так. Алексей развернулся и ушёл, не сказав больше ни слова.

Дома он забаррикадировал дверь стулом, проверил окна, спрятал нож под подушкой. Сон больше не приходил~-- он сидел в темноте, слушая каждый шорох. Доверие, которое он пытался построить, рухнуло. Если даже старушка~-- часть этого кошмара, то кому ещё можно доверять?

\chapter{Последний рубеж}

Алексей сидел на кухне, обхватив голову руками. Нож, который он прошлой ночью положил под подушку, теперь лежал перед ним на столе, тускло блестя в утреннем свете. Барон ходил кругами, поскуливая, словно чувствовал, что хозяин на грани. Нападение в спальне, пуговица, улыбка соседки~-- всё крутилось в его голове, как заевшая пластинка. Но он решил: хватит быть жертвой.

%«Я не дам им победить»,~-- сказал он вслух, глядя на Барона.~-- «Мы будем драться.»
\begin{dialogue}
    «-- Я не дам им победить», -- сказал он вслух, глядя на Барона. -- «Мы будем драться.»
\end{dialogue}

Первым делом он поехал в городок. В местном магазине электроники он купил две дешёвые камеры видеонаблюдения и простой рекордер. Продавец, пожилой мужчина с густыми усами, посмотрел на него с любопытством.

\begin{dialogue}
«-- Что, воры завелись?»
\end{dialogue}

\begin{dialogue}
«-- Типа того»,~-- буркнул Алексей, не вдаваясь в детали.
\end{dialogue}

Дома он потратил полдня, устанавливая камеры: одну у входной двери, другую во дворе, направив на окна. Подключил их к ноутбуку, настроил запись. Теперь каждый шорох, каждый шаг будет зафиксирован. Он даже скачал приложение, чтобы следить за камерами с телефона. Впервые за неделю он почувствовал, что у него есть хоть какой-то контроль.

Потом он открыл браузер и начал искать. Если кто-то играет с ним в эти игры, он не первый. На форумах он нашёл темы про странные записки, ночные вторжения, чувство слежки. Люди писали о «теневых группах», о «коллективном давлении», но всё звучало как теории заговора. Один пользователь, под ником \textit{ShadowWatcher}, оставил пост:

\begin{quote}
«Они приходят, когда ты слаб. Не доверяй никому.»
\end{quote}

Алексей написал ему в личку, надеясь на ответ.

К вечеру он почти поверил, что сделал шаг вперёд. Камеры работали, форум давал надежду, а Барон мирно дремал у его ног. Он даже приготовил себе ужин~-- макароны с сыром, первый нормальный приём пищи за несколько дней. Сидя за столом, он смотрел на экран ноутбука, где транслировались чёрно-белые кадры двора. Всё было тихо.

Но ночью тишина лопнула. Он проснулся от звука уведомления на телефоне. Камера у двери сработала на движение. Алексей схватил телефон, открыл приложение~-- и замер. На экране была фигура в капюшоне, стоящая прямо перед камерой. Лица не видно, только тёмный силуэт. Фигура подняла руку, показав лист бумаги с надписью:

\begin{quote}
«Ты следующий.»
\end{quote}

Алексей выскочил из кровати, схватил нож и бросился к двери. Барон лаял, как бешеный. Он распахнул дверь~-- никого. Только холодный ветер гулял по двору. Камера всё ещё показывала пустоту, но запись осталась. Он проверил вторую камеру~-- та же фигура мелькнула у окна, а потом исчезла.

Сердце колотилось, но он заставил себя сесть и просмотреть записи. Фигура появлялась и исчезала, как призрак, не оставляя следов. Алексей проверил замки, окна~-- всё было закрыто. Как? Как они это делают?

Утром пришло сообщение с форума. \textit{ShadowWatcher} ответил:

\begin{quote}
«Они знают, что ты ищешь. Беги, пока можешь.»
\end{quote}

Алексей стукнул кулаком по столу. Бежать? Опять? Нет, он устал убегать.

Но кошмар продолжился. Когда он вышел покормить Барона, пса во дворе не оказалось. Ошейник лежал на крыльце, а рядом~-- записка:

\begin{quote}
«Сдавайся.»
\end{quote}

Алексей закричал, его голос разнёсся по пустому двору. Он обыскал каждый угол, звал Барона, но тот пропал.

Вернувшись в дом, он рухнул на диван. Камеры, форум, нож~-- всё бесполезно. Они забрали его пса, его последнего союзника. Ярость захлестнула его. Он схватил стул и швырнул его в стену, потом ещё один. Кричал, пока горло не охрипло. Соседи, наверное, слышали, но ему было плевать.

Сидя среди обломков, он понял: это не просто угрозы. Это война. Они хотят сломать его, и они побеждают. Его руки дрожали, глаза горели от слёз, которых он не мог выплакать. Барон был последней ниточкой, связывавшей его с нормальностью, и теперь она оборвалась.

\chapter{Мнимая победа}

Алексей проснулся на полу среди обломков мебели. Утренний свет пробивался через занавески, освещая хаос, который он устроил прошлой ночью: перевёрнутый стул, расколотая лампа, следы его кулаков на стене. Горло саднило от криков, а в голове гудело, как после шторма. Он поднялся, чувствуя, как каждая мышца протестует, и посмотрел на свои руки~-- костяшки были сбиты в кровь. Пустота после пропажи Барона была невыносимой, но вчерашняя ярость выжгла часть страха. Теперь он знал: нужно что-то делать, иначе он сломается окончательно.

Он умылся холодной водой, глядя в зеркало. Лицо осунулось, под глазами залегли тёмные круги, а взгляд стал диким, как у загнанного зверя. «Я не сдамся»,~-- сказал он своему отражению, сжимая края раковины. Ему нужна была помощь, настоящая, а не призрачные советы с форумов. Полиция~-- последний шанс.

Алексей собрал всё, что у него было: записи с камер, записки, фотографию себя спящего, даже пуговицу Марии Ивановны. Он сложил улики в старый рюкзак, надел куртку и вышел из дома. Улица казалась слишком тихой~-- ни звука машин, ни голосов соседей. Только ветер шуршал, и это нервировало его ещё больше. Он оглянулся, проверяя, нет ли кого за спиной, но двор был пуст. Ошейник Барона всё ещё лежал на крыльце, и Алексей отвёл взгляд, чтобы не сорваться снова.

Местный полицейский участок находился в центре городка, в низком здании с облупившейся краской. Алексей толкнул дверь и вошёл. Внутри пахло кофе и старой бумагой. За стойкой сидел дежурный~-- крепкий мужчина лет сорока с усталым лицом и короткой щетиной.

\begin{dialogue}
«-- Чем могу помочь?»
\end{dialogue}

Алексей выложил рюкзак на стол и начал говорить, стараясь держать голос ровным:

\begin{dialogue}
«-- У меня проблемы. Кто-то преследует меня. Записки, вторжения в дом, нападения. Мой пёс пропал. Вот доказательства.»
\end{dialogue}

Дежурный поднял бровь, отложил газету и взял первую записку~-- «Ты не спрячешься». Он прочитал её, повертел в руках, затем посмотрел на Алексея.

\begin{dialogue}
«-- Серьёзно, что ли? И давно это длится?»
\end{dialogue}

\begin{dialogue}
«-- С тех пор, как я сюда переехал. Недели две. Сначала думал, шутки, но потом\ldots»~-- замялся Алексей,~-- «это уже не шутки.»
\end{dialogue}

Офицер вздохнул, будто такие истории слышал каждый день, но всё же принял остальные улики. Он просмотрел фото, записи с камер, даже повертел пуговицу, хмыкнув.

\begin{dialogue}
«-- Ладно, разберёмся. Напиши заявление, оставь это у нас. Мы проверим.»
\end{dialogue}

Алексей сел за стол в углу, заполняя бланк. Рука дрожала, но он старался писать чётко, излагая всё: от первой записки до пропажи Барона. Дежурный тем временем вызвал коллегу~-- молодого парня в форме, который выглядел так, будто только что закончил училище.

\begin{dialogue}
«-- Слушай, Вить, глянь тут»,~-- сказал дежурный, кивая на улики,~-- «Странное дело, но надо проверить. Может, местные хулиганы разошлись.»
\end{dialogue}

Виктор, второй офицер, кивнул и начал просматривать записи с камер. Алексей закончил заявление и отдал его, чувствуя, как тяжесть в груди чуть ослабла. Они обещали патрулировать его улицу, поговорить с соседями, проверить записи. Впервые за долгое время он почувствовал, что не один.

\begin{dialogue}
«-- Иди домой»,~-- сказал дежурный, возвращая ему пустой рюкзак,~-- «Если что найдём, позвоним. И не паникуй, разберёмся.»
\end{dialogue}

Алексей вышел из участка с лёгким облегчением. Солнце стояло высоко, городок оживал: дети бегали по улице, старушки сидели на лавочках, кто-то тащил сумки из магазина. Он даже остановился у кафе, купил себе кофе и булочку~-- простые вещи, которые вдруг показались почти роскошью. Сидя за столиком на улице, он смотрел на прохожих и думал: «Может, теперь всё закончится.»

Дома он прибрался, выкинул обломки мебели, починил лампу. Впервые за неделю он включил музыку~-- старый рок, который любил ещё в юности,~-- и попытался расслабиться. Он даже достал блокнот и начал записывать свои мысли, как делал раньше, до переезда. «Полиция поможет. Я справлюсь»,~-- написал он, подчёркивая слова.

К вечеру он лёг спать с ножом под подушкой~-- привычка, от которой пока не мог избавиться,~-- но впервые заснул без кошмаров. Ему снилось поле, Барон, бегущий к нему с радостным лаем, и тёплое солнце над головой.

\chapter{Падение в бездну}

Алексей вышел из участка на рассвете. Холодный воздух ударил в лицо, но он едва заметил это. Его отпустили «за недостатком улик»~-- так сказал дежурный, глядя в сторону, будто стесняясь своих слов. Мария Ивановна, оказывается, «перепутала» его с кем-то другим, а синяки и порезы объяснила «падением». Всё это звучало как плохая шутка, но Алексею было уже плевать. Он просто хотел домой.

Улицы городка были пусты, только утренний туман стелился над землёй, приглушая звуки. Он шёл медленно, ноги подкашивались от усталости. Ночь в камере выжала из него последние силы, а шёпот сокамерника~-- «Общество видит всё»~-- всё ещё звучал в голове, как заевшая запись. Он пытался отмахнуться от этих слов, но они цеплялись, как колючки.

Дома было тихо. Разбитая мебель, пустой ошейник Барона на крыльце, камеры, которые не помогли~-- всё напоминало о его поражении. Алексей рухнул на диван, закрыв глаза. Ему нужно было отдохнуть, собраться с мыслями, найти хоть какой-то выход. Он даже позволил себе улыбнуться, вспоминая, как утром в участке ему дали кофе~-- горький, но горячий. Может, это конец кошмара? Может, они оставят его в покое?

Он задремал, и ему приснилось поле~-- то же, что прошлой ночью, только теперь Барон не бежал к нему. Пёс стоял вдалеке, глядя с укором, а вокруг поднимались тени, шепча что-то неразборчивое. Алексей проснулся от звука шагов~-- или ему показалось? Он сел, оглядываясь. Тишина. Только часы тикали на стене, отмеряя секунды.

Решив отвлечься, он вышел в небольшой парк за домом. Там было пусто, только птицы чирикали в кронах деревьев. Алексей сел на скамейку, глядя на небо. Облака плыли медленно, и на миг он почувствовал себя почти живым. 
\begin{innerthought}
    Я справлюсь, -- подумал он, сжимая кулаки.
\end{innerthought}

Но этот покой был обманом.

Он заметил движение в кустах. Сначала подумал~-- ветер, но потом услышал шорох, слишком чёткий, слишком близкий. Алексей встал, напрягая слух. Шаги. Много шагов. Он обернулся~-- и замер.

Из тумана вышли люди. Десятки фигур в масках, безмолвных, как призраки. Они не бежали, не кричали~-- просто шли, окружая его. Маски были белыми, без глаз, только чёрные рты кривились в странных улыбках. Алексей отступил, сердце заколотилось.

\begin{dialogue}
«-- Кто вы? Чего вам надо?»~-- закричал он, но голос дрожал.
\end{dialogue}

Они не ответили. Просто стояли, глядя на него~-- или сквозь него. Он повернулся, чтобы бежать, но они были везде: за деревьями, у выхода из парка, даже на тропинке к дому. Алексей рванулся вперёд, расталкивая их, но руки хватали пустоту~-- фигуры расступались, как дым, и тут же смыкались снова.

Он бежал, пока не споткнулся и не упал на траву. Дыхание сбилось, в горле першило. Подняв голову, он увидел, что они стоят вокруг, ближе, чем раньше. Маски блестели в тусклом свете, и теперь он разглядел: под них не было лиц~-- только тьма.

\begin{dialogue}
«-- Оставьте меня!»~-- закричал он, вставая на колени.~-- «Что я вам сделал?!»
\end{dialogue}

Тишина была ответом. Они не двигались, но их присутствие давило, как бетонная плита. Алексей бил кулаками по земле, кричал, пока голос не сорвался в хрип. Он не знал, сколько прошло времени~-- минуты, часы?~-- но наконец они начали отступать, растворяясь в тумане, как дурной сон.
Он остался один, дрожащий, с мокрыми от слёз щеками. Поднявшись, он побрёл домой, но ноги едва слушались. В голове крутился хаос: кто они? Почему он? Это реальность или его разум уже сдаёт?

Дома он запер дверь, зашторил окна, сел в угол с ножом в руках. Тишина вернулась, но теперь она была врагом. Он ждал, прислушиваясь, пока не услышал стук в дверь. Сердце замерло. Стук повторился~-- громче, настойчивее.

Он подошёл к двери, сжимая нож, и открыл её одним рывком. На пороге стоял мужчина в белом халате, с доброй улыбкой и чемоданчиком в руках. За ним~-- двое в форме, похожие на тех, кто его арестовывал.

\begin{dialogue}
«-- Алексей Кравцов?»
\end{dialogue}

\begin{dialogue}
«-- Я здесь, чтобы помочь»,~-- сказал мужчина мягко.~-- «Сядьте, пожалуйста.»
\end{dialogue}

Его пригласили сесть за стол. Мужчина в халате представился:

\begin{dialogue}
«-- Я доктор Лебедев.»
\end{dialogue}

За ним стояли люди в форме, их лица были серьёзны, но без особых эмоций.

\begin{dialogue}
«-- Мы пришли, чтобы помочь вам»,~-- продолжил доктор Лебедев.~-- «Вы слишком напряжены.»
\end{dialogue}

Алексей пытался сопротивляться, но силы оставляли его. Он смотрел, как доктор Лебедев достаёт шприц, а его глаза начинали мутнеть.

\begin{dialogue}
«-- Это для вашего блага»,~-- сказал доктор, вводя иглу в руку Алексея.~-- «Вы забудете всё, что вас мучает.»
\end{dialogue}

Мир поплыл. Алексей пытался сопротивляться, но веки тяжелели. Последнее, что он увидел,~-- улыбка доктора, такая же фальшивая, как маски в парке. Потом тьма.

\end{document}