\documentclass[12pt,a4paper]{book}
\usepackage{fontspec}
\usepackage{polyglossia}
\setmainfont[
  Path = /usr/share/fonts/truetype/ubuntu/,
  Extension = .ttf,
  UprightFont = *-Regular,
  BoldFont = *-Bold,
  ItalicFont = *-Italic,
  BoldItalicFont = *-BoldItalic
]{Ubuntu}
\setmainlanguage{russian}
\setotherlanguage{english}

\usepackage{microtype}            % Improves typography (e.g., spacing, kerning)

% Page layout
\usepackage[a4paper,margin=2.5cm]{geometry} % Consistent margins

% Graphics and images
\usepackage{graphicx}             % For including images if needed

% Headers and footers
\usepackage{fancyhdr}
\pagestyle{fancy}
\fancyhf{}                        % Clear default headers/footers
\lhead{\nouppercase{\leftmark}}   % Chapter title in header (no uppercase)
\rhead{\thepage}                  % Page number on right

% Hyperlinks and clickable references
\usepackage[hidelinks]{hyperref}  % Hyperlinks without visible boxes

% Custom environments
\newenvironment{dialogue}{\begin{quote}\itshape}{\end{quote}} % Dialogue formatting

% Verse spacing adjustment (optional, for better poetry rendering)
\usepackage{setspace}
\AtBeginEnvironment{verse}{\setstretch{1.2}} % Slightly wider line spacing in verse

% Document metadata
\title{Пробежать свой путь (Run Your Way)}
\author{SummerOfF}
\date{\today}

\begin{document}
\maketitle
\tableofcontents
% Chapters will follow here...

\chapter{Безмолвный наблюдатель}

\begin{verse}
Не лает, не кусается,\\
На кошек не бросается,\\
И на прохожих ноль внимания,\\
Такое вот у нас воспитание.
\end{verse}

Эти слова звучали в голове Алексея каждое утро, как старая пластинка, заезженная до хрипоты. Они были его правилом, его невидимой броней, защищающей от хаоса внешнего мира. Алексей --- высокий, худощавый парень с бледной кожей и темными глазами, в которых, казалось, всегда таилась тень. Его взгляд редко задерживался на чем-то конкретном, скользя куда-то вдаль, словно он искал ответы там, где их не могли найти другие.

\section{Утро как ритуал}

День начинался без сюрпризов. Алексей проснулся в 7:00, как по невидимому сигналу, без звука будильника. Его комната была почти пуста: кровать с серым покрывалом, деревянный стол, стул и полка, заставленная книгами. На стене --- одинокий постер с лабиринтом, чьи черные линии извивались, словно отражение его мыслей. Никаких фотографий, никаких украшений. Только книги --- психология, философия, исследования человеческого разума.

Он натянул привычную одежду: серые брюки, белую рубашку, черный пиджак. Ничего яркого, ничего броского. Алексей не хотел выделяться. Он вообще не хотел, чтобы его замечали.

На улице было шумно: машины гудели, люди спешили, перекрикивались утренние голоса. Но Алексей шел своей дорогой, не поднимая глаз, не отвечая на случайные <<доброе утро>> соседей. Его лицо оставалось спокойным, почти застывшим.

\section{День без волн}

В метро он занял своё место у окна. Смотрел, как мелькают черные стены туннеля, как отражаются лица пассажиров в стекле. Его собственное отражение было бледным пятном, почти призрачным. Он не вмешивался в разговоры, не реагировал на толчки и суету. Просто наблюдал.

На работе --- в небольшой IT-компании --- Алексей был тенью. Отвечал на запросы клиентов, чинил баги, молчал. Коллеги считали его странным, но безвредным. <<Тихий>>, --- говорили они за его спиной. Ему это нравилось. Тишина была его союзником.

В обед он ушёл в парк, сел на скамейку с книгой по социальной психологии. Читал о том, как общество формирует личность, как незаметно подчиняет себе волю. Иногда поднимал глаза, чтобы посмотреть на людей: маму с коляской, старика с газетой, подростков с телефонами. Они казались ему персонажами чужой пьесы.

\section{Трещина в привычном}

Вечером, возвращаясь домой, он заметил неладное. Его улица, обычно тихая, гудела. Толпа людей собралась у одного из домов, их голоса сливались в тревожный гул. Алексей замедлил шаг, но не остановился. Он не любил толпы. Не любил быть частью чего-то.

Проходя мимо, он невольно уловил обрывки фраз:
\begin{dialogue}
\textbf{--- "...такой молодой..."}\\[0.5em]
\textbf{--- "...никто не ожидал..."}\\[0.5em]
\textbf{--- "...как это вообще могло случиться?"}
\end{dialogue}

Его брови слегка дрогнули. Что-то произошло. Но он не стал задерживаться, не стал спрашивать. Просто пошёл дальше, к своему подъезду.

Дома он приготовил ужин --- макароны с сыром, простое и знакомое. Включил телевизор, больше для фона, чем для интереса. Ел медленно, перебирая в голове мысли о дне. Но тут его внимание привлёк репортаж. На экране --- его улица, мигалки полицейских машин, та самая толпа. Диктор сухо сообщил: молодой человек найден мертвым в одном из домов. Предположительно --- самоубийство. Расследование продолжается.

Алексей замер, вилка зависла в воздухе. <<Самоубийство?>> --- На его улице? Он вспомнил лица в толпе, их шёпот. Это был кто-то из соседей? Кто-то, кого он мог видеть каждый день, но не замечал.

\section{Ночь вопросов}

Он выключил телевизор и подошёл к окну. Улица опустела. Ни машин, ни людей --- только фонари бросали желтые пятна на асфальт. Но в груди Алексея росло странное чувство. Тревога? Любопытство? Он не мог понять.

Сев за ноутбук, он начал искать новости. Нашёл пару статей, но они были скупы на детали: <<молодой человек>>, <<обнаружен соседями>>, <<полиция выясняет>>. Ничего конкретного. Алексей закрыл крышку и откинулся на стуле. Почему это его зацепило? Он ведь всегда проходил мимо чужих жизней, не касаясь их.

Его взгляд упал на полку. Он взял книгу --- <<Психология толпы>> Гюстава Лебона. Открыл наугад и прочитал:
\begin{quote}
Толпа --- это сила, лишённая разума. Она подчиняется импульсам, следует за невидимыми нитями страха или восторга.
\end{quote}

Толпа на улице. Что ими двигало? Жажда сплетен? Скорбь? Или что-то тёмное, скрытое под масками обыденности?

\section{Шаг к пропасти}

Алексей лёг в кровать, но сон не шел. Его мысли кружились вокруг случившегося. Он вспомнил бабушку, которая воспитала его после смерти родителей.
\begin{quote}
Не вмешивайся, Алексей. Не привлекай внимания. Живи тихо.
\end{quote}
Эти слова он впитал с детства, как закон.

Но сегодня закон дал трещину. Впервые он почувствовал, что не может просто отвернуться. Что-то тянуло его узнать больше. Завтра он попробует поговорить с соседями. Или хотя бы найдёт больше информации.

Закрыв глаза, он пытался уснуть. Но в темноте перед ним вставали образы: толпа, мигалки, чьё-то невидимое лицо. Он не знал, что это начало. Не знал, что его тихая жизнь скоро рухнёт, как карточный домик, под напором сил, которые он не сможет контролировать.

\chapter{Точка разлома}

Алексей сидел в своей комнате, уставившись на пустую страницу ноутбука. Тишина давила, но это была его тишина --- знакомая, привычная, как старый свитер. Он только что закончил очередной рабочий день, тихо устраняя баги в коде для компании, где никто не знал его настоящего имени. <<Тихий>> --- так его называли коллеги, и ему это нравилось. Чем меньше внимания, тем лучше.

Но сегодня что-то было иначе. После того случая с самоубийством на его улице --- парень, которого нашли в соседнем доме --- тишина стала другой. Она больше не успокаивала, а звенела в ушах, как натянутая струна. Алексей встал и подошёл к окну. Улица была пуста, только фонари мигали в ночи, бросая длинные тени на асфальт.

Он вспомнил, как утром спустился к соседям. Решив узнать больше и нарушить своё правило невмешательства, он постучался в дверь. Её открыла пожилая женщина с усталыми глазами.

\begin{dialogue}
\textbf{--- Вы слышали про парня?} --- спросил он, стараясь говорить нейтрально.
\end{dialogue}

Женщина кивнула, её взгляд скользнул по его лицу, будто изучая.

\begin{dialogue}
\textbf{--- Бедняга. Никто не ожидал. Говорят, он был странным... замкнутым. А потом просто... сломался.}
\end{dialogue}

\begin{dialogue}
\textbf{--- Сломался?} --- переспросил Алексей, чувствуя холодок.
\end{dialogue}

\begin{dialogue}
\textbf{--- Да. Люди судачили, что он не выдержал. Знаете, как это бывает --- все смотрят, все шепчутся. А он... он не смог.}
\end{dialogue}

Она замолчала, а Алексей почувствовал, как его сердце сжалось. Он поблагодарил её и ушёл, но слова женщины --- <<Все смотрят, все шепчутся>> --- застряли в голове. Он знал это чувство, знал, как взгляды чужих людей могут обжигать, как слова, сказанные за спиной, проникают под кожу.

Вернувшись домой, он включил телевизор, чтобы заглушить мысли. Репортаж о том же самоубийстве всё ещё крутился в новостях. Журналист говорил о <<давлении общества>>, о том, как парень оставил записку: <<Я больше не могу их слышать.>> Алексей выключил экран, его пальцы дрожали. Он вспомнил, как сам иногда ощущал эти взгляды --- на работе, в метро, даже среди соседей. Они были повсюду, как невидимая сеть, затягивающая всё сильнее.

На следующий день он решил прогуляться, чтобы проветрить голову. Улицы оставались такими же: шумными, суетливыми, полными людей, которые либо не замечали его, либо, наоборот, уделяли слишком много внимания. Он прошёл мимо кафе, где группа подростков громко смеялась, бросая взгляды в его сторону, и мимо старика, который что-то бормотал, глядя ему вслед. Каждый звук, каждый шорох казался громче обычного.

Когда он вернулся домой, его встретил сюрприз. На дверной ручке висела записка, написанная кривым почерком:
\begin{quote}
<<Ты следующий.>>
\end{quote}
Алексей замер, оглядываясь. Улица была пуста. Он сорвал записку и скомкал её, но слова уже глубоко врезались в память. Это было не шуткой --- это было предупреждение.

Ночью он почти не спал. Лёжа в темноте, он прислушивался к звукам за окном: шаги? Шепот? Или просто ветер? Его разум кружился, выстраивая цепочки: записка, самоубийство, слова соседки --- всё складывалось в мрачную картину. Он чувствовал, как что-то невидимое сжимается вокруг него, становясь реальностью.

Утром Алексей принял решение. Этот город, эти люди, эта жизнь --- всё это медленно убивало его, как того парня. Он больше не мог оставаться тенью, растворённой под чужими взглядами. Ему нужно было бежать, начать заново, там, где никто его не знает, где он сможет дышать свободно.

Алексей открыл ноутбук и начал искать. Через час он нашёл объявление: небольшой дом в тихом городке у реки, вдали от больших дорог и любопытных глаз. Цена была смешной, а фотографии обещали покой. Он забронировал дом, не раздумывая.

Собирая вещи, он чувствовал странное облегчение.
\begin{quote}
<<Я уйду от всего этого,>> --- говорил он себе, укладывая книги в коробку. <<Найду место, где смогу быть собой.>>
\end{quote}
Но в глубине души он знал, что от себя не убежать. Что-то уже треснуло внутри него, и эта трещина будет только расти.

Когда он закрыл дверь квартиры в последний раз, он бросил взгляд на улицу. Ему показалось, что в тени дома напротив кто-то стоит и наблюдает за ним. Но он не стал оборачиваться. Сев в машину, он включил двигатель и поехал прочь, оставляя позади город, который начал его ломать.

\chapter{Тихая гавань}

\begin{verse}
Не лает, не кусается,\\
На кошек не бросается,\\
И на прохожих ноль внимания,\\
Такое вот у нас воспитание.
\end{verse}

Эти слова эхом отдавались в голове Алексея, когда он медленно шёл по узкой тропинке, ведущей к его новому дому. Маленький городок, затерянный среди зелёных холмов, казался идеальным местом, чтобы начать всё с чистого листа. После всего, что случилось в прошлом --- о чём он старался не думать, --- ему нужно было именно это: тишина, покой и возможность заново обрести себя.

Алексей остановился у ворот своего нового жилища --- скромного, но уютного дома с деревянной верандой и видом на реку. Воздух здесь был свежим, напоенным запахом сосен и влажной земли. Он глубоко вдохнул, ощущая, как напряжение последних месяцев начинает понемногу отпускать.

\begin{dialogue}
\textbf{--- <<Вот и всё,>>} --- пробормотал он, обращаясь к самому себе. --- <<Новая жизнь.>>
\end{dialogue}

Он толкнул калитку и вошёл во двор. Дом был таким, каким его описывали в объявлении: два этажа, белые ставни, небольшая лужайка с цветущими кустами. Алексей улыбнулся, представив, как будет сидеть на веранде с чашкой кофе, слушая пение птиц.

Внезапно из-за угла дома выскочил пёс --- крупный, лохматый, с добрыми глазами. Он подбежал к Алексею, виляя хвостом, и ткнулся носом в его руку.

\begin{dialogue}
\textbf{--- Привет, дружище, ---} сказал Алексей, погладив пса по голове. --- <<Ты, наверное, местный житель?>>
\end{dialogue}

Пёс лизнул его ладонь и сел рядом, словно они были старыми друзьями. Алексей рассмеялся --- впервые за долгое время.

\begin{dialogue}
\textbf{--- Ладно, будешь моим соседом. Или, может, даже компаньоном.}
\end{dialogue}

Он вошёл в дом, оглядываясь. Внутри было чисто и аккуратно: деревянная мебель, светлые занавески, несколько картин на стенах. Всё дышало спокойствием. Алексей поставил чемодан у двери и прошёлся по комнатам, осматривая своё новое пристанище.

В спальне на втором этаже он обнаружил старый письменный стол, заваленный бумагами. Видимо, предыдущий жилец был писателем или журналистом. Алексей подошёл ближе и увидел стопку листов, исписанных от руки. На верхнем листе крупными буквами было выведено: <<Дневник>>.

Он колебался лишь мгновение, прежде чем взять его в руки.
\begin{dialogue}
\textbf{--- <<Может, это поможет мне лучше понять это место,>>} --- подумал он, открывая первую страницу.
\end{dialogue}

\begin{quote}
<<День первый. Я приехал сюда, чтобы забыть. Чтобы исцелиться. Но что-то здесь не так. Люди... они смотрят на меня странно. Как будто знают что-то, чего не знаю я.>>
\end{quote}

Алексей нахмурился. Строки были написаны торопливым почерком, с помарками. Он перевернул страницу.
\begin{quote}
<<День третий. Сегодня ко мне приходила соседка. Принесла пирог. Она улыбалась, но её глаза... они были пустыми. Как у куклы. Я не могу избавиться от ощущения, что за мной наблюдают.>>
\end{quote}

Сердце Алексея забилось чаще. Он отложил дневник, чувствуя, как по спине пробежал холодок. <<Наверное, просто параноик,>> --- подумал он, пытаясь успокоиться.

Он спустился на кухню, чтобы заварить чай. Пока вода грелась, он выглянул в окно. Солнце уже садилось, окрашивая небо в багровые тона. На улице было тихо, лишь изредка доносился лай собак.

Внезапно он заметил движение у забора. Присмотревшись, он увидел фигуру --- человека в тёмной одежде, стоящего неподвижно и, кажется, смотрящего прямо на его дом. Алексей замер, не зная, что делать. Фигура постояла ещё несколько секунд, а затем быстро скрылась за деревьями.

\begin{dialogue}
\textbf{--- Чёрт,} --- прошептал он, отходя от окна.
\end{dialogue}

Сердце колотилось, он налил чай и сел за стол, стараясь сосредоточиться на приятных мыслях. Завтра он обустроит дом, может, даже заведёт огород. И обязательно купит блокнот для собственного дневника --- чтобы записывать свои мысли и следить за прогрессом.

Но когда он лёг спать, сон не шел. Он ворочался, прислушиваясь к каждому шороху. В какой-то момент ему показалось, что он слышит шаги за дверью, но, собравшись с духом, он встал и проверил --- никого не было.

\chapter{Светлый день}

Алексей проснулся рано утром. Солнце только начинало пробиваться сквозь занавески, заливая комнату мягким золотистым светом. Он потянулся, чувствуя себя отдохнувшим, несмотря на тревожные мысли, которые одолевали его накануне вечером. <<Сегодня будет хороший день,>> --- сказал он себе, поднимаясь с кровати.

Спустившись на кухню, он заварил кофе и вышел на веранду. Его новый спутник --- пёс, которого он решил назвать Барон, --- уже ждал его, радостно виляя хвостом. Алексей улыбнулся и потрепал его по голове.

\begin{dialogue}
\textbf{--- Доброе утро, дружище. Готов к новому дню?}
\end{dialogue}

Барон коротко тявкнул, словно соглашаясь, и Алексей рассмеялся. Он присел на ступеньки веранды, вдыхая свежий утренний воздух. Вдалеке журчала река, а вокруг раздавалось пение птиц. Это было именно то, чего он искал, переезжая сюда --- покой и уединение.

После завтрака Алексей решил прогуляться по городку. Ему хотелось получше узнать место, куда он переехал, а заодно познакомиться с соседями. Надев лёгкую куртку, он вышел на улицу.

Городок оказался небольшим, но удивительно уютным. Чистые улочки, ухоженные дома и дружелюбные лица прохожих создавали ощущение тепла. Проходя мимо продуктового магазина, где он вчера делал покупки, Алексей заметил Мишу, владельца, который приветливо махнул ему рукой.

\begin{dialogue}
\textbf{--- Привет, Алексей! Как дела?}
\end{dialogue}

\begin{dialogue}
\textbf{--- Привет, Миша. Всё хорошо, спасибо. Просто гуляю, осматриваюсь.}
\end{dialogue}

\begin{dialogue}
\textbf{--- Отлично! Если что нужно, заходи, не стесняйся.}
\end{dialogue}

Алексей кивнул и продолжил прогулку. Люди на улице здоровались с ним, улыбались, некоторые даже останавливались, чтобы обменяться парой слов. Он познакомился с местным пекарем, который угостил его свежим хлебом, и с библиотекарем, которая пригласила заглянуть в библиотеку.

К середине дня Алексей почувствовал, что его тревоги начинают отступать. <<Может, я зря волновался,>> --- подумал он, вернувшись домой и устроившись с чашкой чая за столом. <<Люди здесь добрые, и мне здесь нравится.>>

Но вечером, когда он готовил ужин, его мысли вновь вернулись к той странной записке, которую он нашёл вчера. Не в силах избавиться от беспокойства, он решил проверить дом. Алексей обошёл все комнаты, проверил окна и двери --- всё было заперто. Однако, войдя в спальню, он замер. На кровати лежала ещё одна записка, написанная тем же неровным почерком: <<Ты не спрячешься>>.

Его руки задрожали. Схватив записку, он скомкал её и швырнул в угол комнаты. <<Кто это делает? Зачем?>> --- выкрикнул он в пустоту. Барон, почуяв его волнение, заскулил и прижался к его ноге.

\begin{dialogue}
\textbf{--- Всё в порядке, дружище,} --- сказал Алексей, стараясь успокоить себя, чем собаку. --- <<Может, это просто детская шалость,>> --- подумал он, хотя в глубине души понимал, что это не так.
\end{dialogue}

Он решил позвонить в полицию. Набрав номер местного участка, Алексей рассказал о записках. Дежурный офицер выслушал его и ответил спокойным голосом:

\begin{dialogue}
\textbf{--- Не волнуйтесь, мы разберёмся. Но, скорее всего, это просто чья-то глупая шутка.}
\end{dialogue}

Повесив трубку, Алексей почувствовал небольшое облегчение. Он решил быть начеку, но не позволить этому испортить его новую жизнь.

\chapter{Остров спасения}

Утро началось с солнца, льющегося через окно прямо на лицо Алексея. Он открыл глаза, чувствуя, как тепло растекается по телу. Впервые за несколько дней он выспался. Записки, тревоги, ночные шорохи --- всё это казалось далёким сном. Барон, свернувшись калачиком у кровати, тихо посапывал.

\begin{dialogue}
\textbf{--- Пора встряхнуться,} --- сказал Алексей, вставая.
\end{dialogue}

Он решил, что хватит сидеть в четырёх стенах и ждать, пока страх сожрёт его изнутри. Сегодня он попробует жить, как нормальный человек.

После завтрака он позвонил Мише, парню из магазина, с которым успел обменяться парой слов накануне.

\begin{dialogue}
\textbf{--- Привет, Миш. Не хочешь на рыбалку? У реки вроде красиво, да и погода отличная.}
\end{dialogue}

\begin{dialogue}
\textbf{--- Давай! --- Бери удочки, если есть, а я захвачу пиво. Встретимся у моста через час.}
\end{dialogue}

Алексей улыбнулся. Может, это и есть то, ради чего он сюда приехал --- простые радости, дружеская компания, возможность дышать полной грудью.

Он собрал небольшой рюкзак: удочку, найденную в сарае, бутерброды, бутылку воды. Барон проводил его до двери, но Алексей решил оставить пса дома --- на рыбалке тот мог бы спугнуть улов.

У реки было тихо. Вода блестела под солнцем, лёгкий ветерок шевелил листву. Миша уже ждал у моста, держа в руках пару банок пива и старый рюкзак.

\begin{dialogue}
\textbf{--- Ну что, новичок, готов поймать свою первую рыбу?}
\end{dialogue}

\begin{dialogue}
\textbf{--- Если только она сама на крючок прыгнет,} --- пошутил Алексей, и оба рассмеялись.
\end{dialogue}

Они нашли укромное место на берегу, где река делала изгиб, и разложили снасти. Миша оказался весёлым собеседником: травил байки про местных жителей, рассказывал, как однажды поймал щуку размером с руку, и даже спел пару строчек из какой-то дурацкой песни. Алексей расслабился, чувствуя, как напряжение последних дней уходит. Они пили пиво, шутили, а удочки покачивались в воде, словно соглашаясь с их беззаботностью.

\begin{dialogue}
\textbf{--- Знаешь, ---} сказал Миша, глядя на реку, --- \textbf{--- тут хорошо. Спокойно. Иногда кажется, что весь мир где-то там, а мы тут в своём маленьком раю.}
\end{dialogue}

\begin{dialogue}
\textbf{--- Да, ---} кивнул Алексей, --- \textbf{--- Именно этого мне и не хватало.}
\end{dialogue}

Солнце поднялось выше, и они решили перебраться на лодку, которую Миша притащил из кустов. Это была старая деревянная посудина, но выглядела крепкой. Они оттолкнулись от берега, и река мягко понесла их вниз по течению. Алексей наблюдал, как деревья проплывают мимо, как отражаются облака в воде, и впервые за долгое время чувствовал себя свободным.

Но к вечеру небо начало темнеть. Облака сгустились, ветер усилился, и река, ещё недавно спокойная, заволновалась. Миша нахмурился.

\begin{dialogue}
\textbf{--- Похоже, буря идёт. Надо возвращаться.}
\end{dialogue}

Они начали грести к берегу, но волны становились всё выше. Лодка качалась, вода плескалась через борта. Алексей сжал вёсла, стараясь держать ритм, но сердце уже билось быстрее.

\begin{dialogue}
\textbf{--- Держись!} --- крикнул Миша, когда очередной порыв ветра накренил лодку.
\end{dialogue}

И тут раздался треск. Днище треснуло, и холодная вода хлынула внутрь. Алексей бросил вёсла, хватаясь за борта, но лодка стремительно тонула. Они оказались в воде, среди темноты и воющего ветра.

\begin{dialogue}
\textbf{--- Миша!} --- крикнул Алексей, пытаясь разглядеть друга в хаосе волн.
\end{dialogue}

Он увидел его в нескольких метрах --- Миша барахтался, но его лицо... Алексей замер. Миша смотрел на него с улыбкой: не с паникой, не с отчаянием, а с какой-то странной, почти зловещей улыбкой. А потом волна накрыла его, и он исчез.

Алексей боролся с течением, лёгкие горели, руки немели от холода. Он не знал, сколько времени прошло, но наконец его выбросило на берег. Кашляя и задыхаясь, он рухнул на мокрую траву. Дождь хлестал по лицу, молнии освещали небо.

Он лежал, пытаясь отдышаться, когда заметил что-то в кармане куртки. Дрожащими руками он вытащил мокрый клочок бумаги. На нём размытыми чернилами было написано:
\begin{quote}
<<Мы ближе, чем ты думаешь.>>
\end{quote}

Алексей отбросил записку, его разум закружился. Миша? Это он? Или кто-то другой? Дождь смывал грязь с его рук, но не мог смыть страх, который теперь жил внутри него. Он поднялся и побрёл домой, оставляя за собой реку, которая чуть не стала его могилой.

Когда он дошёл до дома, Барон встретил его тревожным лаем. Алексей рухнул на пол, прижимая пса к себе. Его тело дрожало, а в груди росло ощущение, что даже здесь, в этом <<раю>>, он не один. Кто-то --- или что-то --- следило за ним. И этот кто-то знал, как добраться до него.

\chapter{Тепло очага}

Алексей проснулся от лая Барона. Солнечный свет пробивался через щели в занавесках, но он чувствовал себя так, будто не спал вовсе. Ночь после рыбалки была полна кошмаров: тёмная вода, ухмылка Миши, записка, растворяющаяся в реке. Он сел на кровати, потирая виски. Тело ломило от вчерашнего купания в холодных волнах, но хуже всего было внутри --- страх, как ржавчина, разъедал его изнутри.

\begin{dialogue}
\textbf{--- Хватит,} --- сказал он вслух, глядя на Барона, который тревожно смотрел на него. --- \textbf{Я не дам этому разрушить меня.}
\end{dialogue}

Он решил взять всё в свои руки. Этот дом должен стать его убежищем, а не ловушкой. После завтрака --- чёрный кофе и кусок хлеба, больше он не смог в себя впихнуть --- Алексей принялся за дело. Он проверил все замки на дверях и окнах, подкрутил шурупы там, где они болтались. В сарае он нашёл старую доску и забил ею окно на заднем дворе, которое казалось ему слишком уязвимым. Каждый стук молотка звучал как вызов: <<Я не сдамся.>>

К полудню дом выглядел крепче. Алексей даже позволил себе улыбнуться, глядя на Барона, который носился по двору, гоняясь за бабочкой.

\begin{dialogue}
\textbf{--- Вот так, дружище. Теперь это наша крепость.}
\end{dialogue}

Он включил старый радиоприёмник, найденный на кухне, и настроил его на местную станцию. Тихая музыка заполнила дом, создавая иллюзию уюта. Алексей сел на диван с чашкой чая, чувствуя, как напряжение понемногу отпускает. Он даже начал думать, что буря на реке --- случайность, а записка --- чья-то дурацкая шутка. Может, Миша просто утонул, а улыбка была плодом его воображения?

Раздался стук в дверь. Алексей вздрогнул, пролив чай на колени. Барон насторожился, но не залаял --- просто смотрел на дверь, чуть склонив голову.

\begin{dialogue}
\textbf{--- Кто там?} --- крикнул Алексей, не вставая.
\end{dialogue}

\begin{dialogue}
\textbf{--- Это я, Мария Ивановна!} --- послышался знакомый голос соседки. --- \textbf{Принёсла тебе кое-что, сынок.}
\end{dialogue}

Он выдохнул, чувствуя себя немного глупо за свою реакцию. Открыв дверь, он увидел старушку с корзинкой в руках. Она улыбалась, её морщинистое лицо выглядело добрым и спокойным.

\begin{dialogue}
\textbf{--- Вот, травяной чай,} --- сказала она, протягивая корзинку. --- \textbf{Успокаивает нервы. Ты вчера выглядел таким бледным, подумала, тебе пригодится.}
\end{dialogue}

\begin{dialogue}
\textbf{--- Спасибо,} --- ответил Алексей, принимая подарок. --- \textbf{Заходите, если хотите.}
\end{dialogue}

Мария Ивановна покачала головой.

\begin{dialogue}
\textbf{--- Нет, сынок, дела зовут. Но ты пей чай, отдыхай. И заходи, если что.}
\end{dialogue}

Она ушла, а Алексей закрыл дверь, чувствуя тепло от её заботы. Он заварил чай --- пахло мятой и чем-то ещё, терпким, но приятным. Сделав глоток, он устроился на диване, глядя, как Барон играет с костью у порога. Впервые за долгое время он почувствовал себя в безопасности.

К вечеру чай подействовал --- веки стали тяжёлыми, тело расслабилось. Алексей лёг в кровать, даже не выключив свет на кухне. Сон пришёл быстро, глубокий и тёмный, как колодец.

Ему снилось, что он стоит посреди комнаты, а вокруг --- люди в масках. Они не двигались, просто смотрели, их глаза блестели в темноте. Он пытался кричать, но голос пропал. Внезапно маски начали падать, и под ними были знакомые лица: Миша, соседка, даже продавец из магазина. Они улыбались, но не так, как живые люди, а как манекены.

Алексей проснулся с криком, хватая воздух. Комната была тёмной, свет на кухне давно погас. Он рванулся к выключателю, но остановился. Что-то было не так. Воздух казался тяжёлым, а тишина --- слишком густой.

Барон зарычал, глядя на дверь спальни. Алексей медленно подошёл к ней, сердце колотилось так, что отдавалось в ушах. Он открыл дверь --- и замер. На столе в гостиной лежала фотография. Он сам, спящий на диване, с закрытыми глазами и чуть открытым ртом. Снимок был сделан этой ночью.

\begin{dialogue}
\textbf{--- Нет... нет, нет, нет,} --- прошептал он, отступая назад.
\end{dialogue}

Барон залаял, бросаясь к окну, но там никого не было. Алексей схватил телефон, чтобы позвонить в полицию, но экран мигнул и погас --- батарея села, хотя утром была полной.

Он рухнул на стул, сжимая голову руками. Фотография лежала перед ним, как доказательство: кто-то был здесь, внутри его <<крепости>>, пока он спал. Чай? Неужели старушка... Нет, это бред. Или не бред?

Барон продолжал рычать, и Алексей понял: покой --- это иллюзия. Кто-то проник в его дом, в его разум, и теперь он не знал, где реальность, а где его собственные страхи.

\chapter{Дружеская поддержка}

Утро встретило Алексея холодом и тишиной. Он сидел за столом, уставившись на фотографию, которая всю ночь лежала перед ним, как немой укор. Барон спал у его ног, но даже присутствие пса не могло прогнать страх, который пустил корни в его груди. Он не спал, не ел --- просто смотрел, пытаясь понять, кто и как это сделал.

<<Я не могу так жить,>> --- наконец сказал он себе, вставая. --- Нужно было действовать, искать помощь. Один он с этим не справится.

После чашки кофе, которая не согрела, а только обожгла горло, Алексей решил пойти к Марии Ивановне. Её чай, её доброта --- всё это теперь казалось подозрительным, но она была единственным человеком, с кем он хоть немного сблизился. Может, он ошибся, и она действительно просто милая старушка? Ему нужно было верить хоть кому-то.

Он постучал в её дверь, сжимая кулаки, чтобы унять дрожь. Мария Ивановна открыла почти сразу, и её лицо осветилось улыбкой.

\begin{dialogue}
\textbf{--- Ой, Алексей, заходи! Что-то ты бледный опять. Не заболел?}
\end{dialogue}

\begin{dialogue}
\textbf{--- Нет,} --- ответил он, входя в её уютную кухню, --- \textbf{просто... не спал. Хотел поговорить.}
\end{dialogue}

\begin{dialogue}
\textbf{--- Садись, сынок,} --- сказала она, поставив перед ним чашку чая. Однако Алексей отодвинул её, всё ещё помня вчерашний ужас.

\begin{dialogue}
\textbf{--- Мария Ивановна,} --- начал он, стараясь говорить спокойно, --- \textbf{у меня дома... странные вещи творятся. Записки, фотографии. Кто-то был внутри, пока я спал.}
\end{dialogue}

Её глаза расширились, но не от испуга, а от любопытства. Она покачала головой.

\begin{dialogue}
\textbf{--- Боже мой, это ужасно. Ты в полицию ходил?}
\end{dialogue}

\begin{dialogue}
\textbf{--- Пока нет. Думал, может, вы что-то знаете. Видели кого-нибудь около моего дома?}
\end{dialogue}

\begin{dialogue}
\textbf{--- Нет, сынок, ничего такого. Но у нас городок тихий, знаешь... Иногда детишки балуются, а иногда ---}
\end{dialogue}

Она замялась, --- старые истории оживают.

\begin{dialogue}
\textbf{--- Какие истории?}
\end{dialogue}

\begin{dialogue}
\textbf{--- Да ерунда, слухи. Говорят, раньше тут люди пропадали. Но это давно было, не бери в голову.}
\end{dialogue}

Алексей кивнул, но её слова только усилили тревогу. Он рассказал ей про рыбалку, про Мишу, про найденную записку. Мария Ивановна слушала внимательно, иногда качая головой, а потом положила руку на его плечо.

\begin{dialogue}
\textbf{--- Ты, главное, не бойся. Если что, приходи ко мне. Я старая, но глаза у меня зоркие. Пригляжу за твоим домом.}
\end{dialogue}

Он ушёл с лёгким облегчением. Её забота казалась искренней, и он решил дать ей шанс. Может, он правда преувеличивает? Вернувшись домой, Алексей занялся делами: почистил двор, покормил Барона, даже включил музыку, чтобы заглушить тишину. День прошёл спокойно, и к вечеру он почти поверил, что всё налаживается.

Но ночью всё рухнуло.

Он проснулся от странного ощущения --- как будто кто-то давил ему на грудь. Открыв глаза, он увидел тень над собой. Кто-то стоял у кровати, держа подушку. Алексей рванулся, но тень прижала подушку к его лицу. Воздух исчез, он задыхался, бился, хватая руками пустоту. Наконец, он сумел сбросить нападающего, и тот с шумом упал на пол. Тень метнулась к двери и исчезла в темноте.

Алексей вскочил, кашляя и задыхаясь. Барон лаял, бросаясь к окну. Включив свет, Алексей осмотрел комнату. На полу лежала маленькая деревянная пуговица с вырезанным цветком. Он видел такую же на кофте Марии Ивановны.

\begin{dialogue}
\textbf{--- Нет...}
\end{dialogue}

Сжав пуговицу в кулаке, он не мог поверить: это не могло быть правдой. Или могло?

Утром он пошёл к ней снова. Она открыла дверь, как ни в чём не бывало, с той же доброй улыбкой.

\begin{dialogue}
\textbf{--- О, Алексей, опять ты! Чайку?}
\end{dialogue}

\begin{dialogue}
\textbf{--- Нет,} --- резко ответил он, показывая пуговицу, --- \textbf{это ваше?}
\end{dialogue}

Она посмотрела на пуговицу, потом на свою кофту --- действительно, одна пуговица отсутствовала. Её лицо осталось спокойным.

\begin{dialogue}
\textbf{--- Ой, наверное, потеряла где-то. А где ты её нашёл?}
\end{dialogue}

\begin{dialogue}
\textbf{--- У себя в спальне. После того, как кто-то пытался меня задушить.}
\end{dialogue}

Мария Ивановна ахнула, прижимая руку к груди.

\begin{dialogue}
\textbf{--- Господи, что ты такое говоришь? Это не я, сынок, клянусь!}
\end{dialogue}

Он смотрел ей в глаза, не зная, что думать --- врёт ли она или нет. Её голос дрожал, но что-то в её взгляде было не так. Алексей развернулся и ушёл, не сказав больше ни слова.

Дома он забаррикадировал дверь стулом, проверил окна, спрятал нож под подушкой. Сон больше не приходил --- он сидел в темноте, слушая каждый шорох. Доверие, которое он пытался построить, рухнуло. Если даже старушка --- часть этого кошмара, то кому ещё можно доверять?

\chapter{Последний рубеж}

Алексей сидел на кухне, обхватив голову руками. Нож, который он прошлой ночью положил под подушку, теперь лежал перед ним на столе, тускло блестя в утреннем свете. Барон ходил кругами, поскуливая, словно чувствовал, что хозяин на грани. Нападение в спальне, пуговица, улыбка соседки --- всё крутилось в его голове, как заевшая пластинка. Но он решил: хватит быть жертвой.

<<Я не дам им победить,>> --- сказал он вслух, глядя на Барона. --- <<Мы будем драться.>>

Первым делом он поехал в городок. В местном магазине электроники он купил две дешёвые камеры видеонаблюдения и простой рекордер. Продавец, пожилой мужчина с густыми усами, посмотрел на него с любопытством.

\begin{dialogue}
\textbf{--- Что, воры завелись?}
\end{dialogue}

\begin{dialogue}
\textbf{--- Типа того,} --- буркнул Алексей, не вдаваясь в детали.
\end{dialogue}

Дома он потратил полдня, устанавливая камеры: одну у входной двери, другую во дворе, направив на окна. Подключил их к ноутбуку, настроил запись. Теперь каждый шорох, каждый шаг будет зафиксирован. Он даже скачал приложение, чтобы следить за камерами с телефона. Впервые за неделю он почувствовал, что у него есть хоть какой-то контроль.

Потом он открыл браузер и начал искать. Если кто-то играет с ним в эти игры, он не первый. На форумах он нашёл темы про странные записки, ночные вторжения, чувство слежки. Люди писали о <<теневых группах>>, о <<коллективном давлении>>, но всё звучало как теории заговора. Один пользователь, под ником \textit{ShadowWatcher}, оставил пост:

\begin{quote}
<<Они приходят, когда ты слаб. Не доверяй никому.>>
\end{quote}

Алексей написал ему в личку, надеясь на ответ.

К вечеру он почти поверил, что сделал шаг вперёд. Камеры работали, форум давал надежду, а Барон мирно дремал у его ног. Он даже приготовил себе ужин --- макароны с сыром, первый нормальный приём пищи за несколько дней. Сидя за столом, он смотрел на экран ноутбука, где транслировались чёрно-белые кадры двора. Всё было тихо.

Но ночью тишина лопнула. Он проснулся от звука уведомления на телефоне. Камера у двери сработала на движение. Алексей схватил телефон, открыл приложение --- и замер. На экране была фигура в капюшоне, стоящая прямо перед камерой. Лица не видно, только тёмный силуэт. Фигура подняла руку, показав лист бумаги с надписью:

\begin{quote}
<<Ты следующий.>>
\end{quote}

Алексей выскочил из кровати, схватил нож и бросился к двери. Барон лаял, как бешеный. Он распахнул дверь --- никого. Только холодный ветер гулял по двору. Камера всё ещё показывала пустоту, но запись осталась. Он проверил вторую камеру --- та же фигура мелькнула у окна, а потом исчезла.

Сердце колотилось, но он заставил себя сесть и просмотреть записи. Фигура появлялась и исчезала, как призрак, не оставляя следов. Алексей проверил замки, окна --- всё было закрыто. Как? Как они это делают?

Утром пришло сообщение с форума. \textit{ShadowWatcher} ответил:

\begin{quote}
<<Они знают, что ты ищешь. Беги, пока можешь.>>
\end{quote}

Алексей стукнул кулаком по столу. Бежать? Опять? Нет, он устал убегать.

Но кошмар продолжился. Когда он вышел покормить Барона, пса во дворе не оказалось. Ошейник лежал на крыльце, а рядом --- записка:

\begin{quote}
<<Сдавайся.>>
\end{quote}

Алексей закричал, его голос разнёсся по пустому двору. Он обыскал каждый угол, звал Барона, но тот пропал.

Вернувшись в дом, он рухнул на диван. Камеры, форум, нож --- всё бесполезно. Они забрали его пса, его последнего союзника. Ярость захлестнула его. Он схватил стул и швырнул его в стену, потом ещё один. Кричал, пока горло не охрипло. Соседи, наверное, слышали, но ему было плевать.

Сидя среди обломков, он понял: это не просто угрозы. Это война. Они хотят сломать его, и они побеждают. Его руки дрожали, глаза горели от слёз, которых он не мог выплакать. Барон был последней ниточкой, связывавшей его с нормальностью, и теперь она оборвалась.

\chapter{Мнимая победа}

Алексей проснулся на полу среди обломков мебели. Утренний свет пробивался через занавески, освещая хаос, который он устроил прошлой ночью: перевёрнутый стул, расколотая лампа, следы его кулаков на стене. Горло саднило от криков, а в голове гудело, как после шторма. Он поднялся, чувствуя, как каждая мышца протестует, и посмотрел на свои руки --- костяшки были сбиты в кровь. Пустота после пропажи Барона была невыносимой, но вчерашняя ярость выжгла часть страха. Теперь он знал: нужно что-то делать, иначе он сломается окончательно.

Он умылся холодной водой, глядя в зеркало. Лицо осунулось, под глазами залегли тёмные круги, а взгляд стал диким, как у загнанного зверя. <<Я не сдамся,>> --- сказал он своему отражению, сжимая края раковины. Ему нужна была помощь, настоящая, а не призрачные советы с форумов. Полиция --- последний шанс.

Алексей собрал всё, что у него было: записи с камер, записки, фотографию себя спящего, даже пуговицу Марии Ивановны. Он сложил улики в старый рюкзак, надел куртку и вышел из дома. Улица казалась слишком тихой --- ни звука машин, ни голосов соседей. Только ветер шуршал, и это нервировало его ещё больше. Он оглянулся, проверяя, нет ли кого за спиной, но двор был пуст. Ошейник Барона всё ещё лежал на крыльце, и Алексей отвёл взгляд, чтобы не сорваться снова.

Местный полицейский участок находился в центре городка, в низком здании с облупившейся краской. Алексей толкнул дверь и вошёл. Внутри пахло кофе и старой бумагой. За стойкой сидел дежурный --- крепкий мужчина лет сорока с усталым лицом и короткой щетиной.

\begin{dialogue}
\textbf{--- Чем могу помочь?}
\end{dialogue}

Алексей выложил рюкзак на стол и начал говорить, стараясь держать голос ровным:

\begin{dialogue}
\textbf{--- У меня проблемы. Кто-то преследует меня. Записки, вторжения в дом, нападения. Мой пёс пропал. Вот доказательства.}
\end{dialogue}

Дежурный поднял бровь, отложил газету и взял первую записку --- <<Ты не спрячешься>>. Он прочитал её, повертел в руках, затем посмотрел на Алексея.

\begin{dialogue}
\textbf{--- Серьёзно, что ли? И давно это длится?}
\end{dialogue}

\begin{dialogue}
\textbf{--- С тех пор, как я сюда переехал. Недели две. Сначала думал, шутки, но потом...} --- замялся Алексей, --- \textbf{это уже не шутки.}
\end{dialogue}

Офицер вздохнул, будто такие истории слышал каждый день, но всё же принял остальные улики. Он просмотрел фото, записи с камер, даже повертел пуговицу, хмыкнув.

\begin{dialogue}
\textbf{--- Ладно, разберёмся. Напиши заявление, оставь это у нас. Мы проверим.}
\end{dialogue}

Алексей сел за стол в углу, заполняя бланк. Рука дрожала, но он старался писать чётко, излагая всё: от первой записки до пропажи Барона. Дежурный тем временем вызвал коллегу --- молодого парня в форме, который выглядел так, будто только что закончил училище.

\begin{dialogue}
\textbf{--- Слушай, Вить, глянь тут, ---} сказал дежурный, кивая на улики, --- \textbf{Странное дело, но надо проверить. Может, местные хулиганы разошлись.}
\end{dialogue}

Виктор, второй офицер, кивнул и начал просматривать записи с камер. Алексей закончил заявление и отдал его, чувствуя, как тяжесть в груди чуть ослабла. Они обещали патрулировать его улицу, поговорить с соседями, проверить записи. Впервые за долгое время он почувствовал, что не один.

\begin{dialogue}
\textbf{--- Иди домой, ---} сказал дежурный, возвращая ему пустой рюкзак, --- \textbf{Если что найдём, позвоним. И не паникуй, разберёмся.}
\end{dialogue}

Алексей вышел из участка с лёгким облегчением. Солнце стояло высоко, городок оживал: дети бегали по улице, старушки сидели на лавочках, кто-то тащил сумки из магазина. Он даже остановился у кафе, купил себе кофе и булочку --- простые вещи, которые вдруг показались почти роскошью. Сидя за столиком на улице, он смотрел на прохожих и думал: <<Может, теперь всё закончится.>>

Дома он прибрался, выкинул обломки мебели, починил лампу. Впервые за неделю он включил музыку --- старый рок, который любил ещё в юности, --- и попытался расслабиться. Он даже достал блокнот и начал записывать свои мысли, как делал раньше, до переезда. <<Полиция поможет. Я справлюсь,>> --- написал он, подчёркивая слова.

К вечеру он лёг спать с ножом под подушкой --- привычка, от которой пока не мог избавиться, --- но впервые заснул без кошмаров. Ему снилось поле, Барон, бегущий к нему с радостным лаем, и тёплое солнце над головой.

\chapter{Падение в бездну}

Алексей вышел из участка на рассвете. Холодный воздух ударил в лицо, но он едва заметил это. Его отпустили <<за недостатком улик>> --- так сказал дежурный, глядя в сторону, будто стыжаясь своих слов. Мария Ивановна, оказывается, <<перепутала>> его с кем-то другим, а синяки и порезы объяснила <<падением>>. Всё это звучало как плохая шутка, но Алексею было уже плевать. Он просто хотел домой.

Улицы городка были пусты, только утренний туман стелился над землёй, приглушая звуки. Он шёл медленно, ноги подкашивались от усталости. Ночь в камере выжала из него последние силы, а шепот сокамерника --- <<Общество видит всё>> --- всё ещё звучал в голове, как заевшая запись. Он пытался отмахнуться от этих слов, но они цеплялись, как колючки.

Дома было тихо. Разбитая мебель, пустой ошейник Барона на крыльце, камеры, которые не помогли --- всё напоминало о его поражении. Алексей рухнул на диван, закрыв глаза. Ему нужно было отдохнуть, собраться с мыслями, найти хоть какой-то выход. Он даже позволил себе улыбнуться, вспоминая, как утром в участке ему дали кофе --- горький, но горячий. Может, это конец кошмара? Может, они оставят его в покое.

Он задремал, и ему приснилось поле --- то же, что прошлой ночью, только теперь Барон не бежал к нему. Пёс стоял вдалеке, глядя с укором, а вокруг поднимались тени, шепча что-то неразборчивое. Алексей проснулся от звука шагов --- или ему показалось? Он сел, оглядываясь. Тишина. Только часы тикали на стене, отмеряя секунды.

Решив отвлечься, он вышел в небольшой парк за домом. Там было пусто, только птицы чирикали в кронах деревьев. Алексей сел на скамейку, глядя на небо. Облака плыли медленно, и на миг он почувствовал себя почти живым. <<Я справлюсь,>> --- подумал он, сжимая кулаки. <<Они не сломают меня.>>

Но этот покой был обманом.

Он заметил движение в кустах. Сначала подумал --- ветер, но потом услышал шорох, слишком чёткий, слишком близкий. Алексей встал, напрягая слух. Шаги. Много шагов. Он обернулся --- и замер.

Из тумана вышли люди. Десятки фигур в масках, безмолвных, как призраки. Они не бежали, не кричали --- просто шли, окружая его. Маски были белыми, без глаз, только чёрные рты кривились в странных улыбках. Алексей отступил, сердце заколотилось.

<<Кто вы? Чего вам надо?>> --- закричал он, но голос дрожал.

Они не ответили. Просто стояли, глядя на него --- или сквозь него. Он повернулся, чтобы бежать, но они были везде: за деревьями, у выхода из парка, даже на тропинке к дому. Алексей рванулся вперёд, расталкивая их, но руки хватали пустоту --- фигуры расступались, как дым, и тут же смыкались снова.

Он бежал, пока не споткнулся и не упал на траву. Дыхание сбилось, в горле першило. Подняв голову, он увидел, что они стоят вокруг, ближе, чем раньше. Маски блестели в тусклом свете, и теперь он разглядел: под ними не было лиц --- только тьма.

<<Оставьте меня!>> --- закричал он, вставая на колени. --- <<Что я вам сделал?!>>

Тишина была ответом. Они не двигались, но их присутствие давило, как бетонная плита. Алексей бил кулаками по земле, кричал, пока голос не сорвался в хрип. Он не знал, сколько прошло времени --- минуты, часы? --- но наконец они начали отступать, растворяясь в тумане, как дурной сон.

Он остался один, дрожащий, с мокрыми от слёз щеками. Поднявшись, он побрёл домой, но ноги едва слушались. В голове крутился хаос: кто они? Почему он? Это реальность или его разум уже сдаёт?

Дома он запер дверь, зашторил окна, сел в угол с ножом в руках. Тишина вернулась, но теперь она была врагом. Он ждал, прислушиваясь, пока не услышал стук в дверь. Сердце замерло. Стук повторился --- громче, настойчивее.

Он подошёл к двери, сжимая нож, и открыл её одним рывком. На пороге стоял мужчина в белом халате, с доброй улыбкой и чемоданчиком в руках. За ним --- двое в форме, похожие на тех, кто его арестовывал.

\begin{dialogue}
\textbf{--- Алексей Кравцов?}
\end{dialogue}

\begin{dialogue}
\textbf{--- Я здесь, чтобы помочь,} --- сказал мужчина мягко. --- \textbf{Сядьте, пожалуйста.}
\end{dialogue}

За столом его пригласили сесть. Мужчина в халате представился:

\begin{dialogue}
\textbf{--- Я доктор Лебедев.}
\end{dialogue}

За ним стояли люди в форме, их лица были серьёзны, но без особых эмоций.

\begin{dialogue}
\textbf{--- Мы пришли, чтобы помочь вам,} --- продолжил доктор Лебедев. --- \textbf{Вы слишком напряжены.}
\end{dialogue}

Алексей пытался сопротивляться, но силы оставляли его. Он смотрел, как доктор Лебедев достаёт шприц, а его глаза начинали мутнеть.

\begin{dialogue}
\textbf{--- Это для вашего блага,} --- сказал доктор, вводя иглу в руку Алексея. --- \textbf{Вы забудете всё, что вас мучает.}
\end{dialogue}

Мир поплыл. Алексей пытался сопротивляться, но веки тяжелели. Последнее, что он увидел, --- улыбка доктора, такая же фальшивая, как маски в парке. Потом тьма.

\end{document}